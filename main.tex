%%%%%%%%%%%%%%%%%%%%%%%%%%%%%%%%%%%%%%%%%%%%%%%%%%%%%%%%%%%%%%%%%%%%%%%%%%%%%%%%
%2345678901234567890123456789012345678901234567890123456789012345678901234567890
%        1         2         3         4         5         6         7         8

\documentclass[a4paper, 10pt, conference]{docstyle/ieeeconf}      % Use this line for a4 paper

\IEEEoverridecommandlockouts                              % This command is only needed if
                                                          % you want to use the \thanks command

\overrideIEEEmargins                                      % Needed to meet printer requirements.

% See the \addtolength command later in the file to balance the column lengths
% on the last page of the document

% The following packages can be found on http:\\www.ctan.org
\usepackage{graphics} % for pdf, bitmapped graphics files
\usepackage{epsfig} % for postscript graphics files
\usepackage{mathptmx} % assumes new font selection scheme installed
\usepackage{times} % assumes new font selection scheme installed
\usepackage{amsmath} % assumes amsmath package installed
\usepackage{amssymb}  % assumes amsmath package installed

% Remove unofficial libraries
\usepackage{subcaption} % allows subfigure
\usepackage{todonotes} % todo notes for prototyping
\usepackage{listings}
\usepackage{algorithm}% http://ctan.org/pkg/algorithms
\usepackage{algpseudocode}% http://ctan.org/pkg/algorithmicx

% External bibliography --- For submittion include citation in file
\usepackage[
backend=biber,
style=ieee,
sorting=nty
]{biblatex}
\addbibresource{bibliography.bib}

\title{\LARGE{ \bf{Avoidance of Static and Moving Obstacles}} \\
Design and Simulation of Obstacle Avoidance Algorithms based on Dynamical Systems
}

\author{Lukas Huber$^{}$ $\cdot$ Aude Billard$^{}$ $\cdot$ Leonardo Urbano$^{}$  % <-this % stops a space
%\thanks{*This work was not supported by any organization}% <-this % stops a space
  \thanks{$^{}$ Learning Algorithms and Systems Laboratory, Swiss Federal Department of Technology in Lausanne (EPFL), Lausanne 1000, Switzerland
 \tt \small {lukas.huber@epfl.ch}
 \tt\small{aude.billard@epfl.ch;leonardo.urbano@epfl.ch}}
}

%\pagenumbering{arabic}

\begin{document}
\maketitle
%\thispagestyle{empty}
%\pagestyle{empty}

%%%%%%%%%%%%%%%%%%%%%%%%%%%%%%%%%%%%%%%%%%%%%%%%%%%%%%%%%%%%%%%%%%%%%%%%%%%%%%%%
\begin{abstract}
  This report evaluates different obstacle avoidance algorithms based dynamical systems (DS). Specifically, this report focuses on a linear DS which are among many other used for pick and place tasks in robotics. It introduces the modulation of DS and observes an existing algorithm which uses a dynamic modulation matrix (DMM). The later algorithm shows limitation with obstacles which are more complex than circles, where it has the tendency to converge to local minima on the obstacle's boundary rather than to converge to the attractor of the DS. \\
The new algorithm is taking advantage of a space transformation inspired by fluid dynamics (IFD) to avoid more complex obstacles. In such a way, impenetrability for any ellipse shaped obstacle can be proofed. Furthermore, convergence to the attractor of the DS can be shown in linear DS which is a large advantage over the DMM.
  % Two further algorithms are proposed and analysed:  obstacle avoidance inspired by fluid dynamics (IFD) and obstacle avoidance with local rotation of space (LRS). These three algorithms are compared in simulation by applying them to linear DS with a single attractor or DS with a limit cycle, static, rotating and moving obstacles are then placed in the DS. %Furthermore, numerical metrics are introduced to quantitatively evaluate the performance of the algorithms. \\
%The IFD and DMM perform similarly based on computational cost and convergence distance. The IFD shows slightly better performance in therms of convergence energy and the IFD changes the DS less. The main disadvantage of the DMM is, that it often creates a local minimum on an ellipse boundary in a linear DS with a single attractor, while the IFD only creates a saddle point on the obstacle's boundary. The LRS is fundamentally different from the other algorithms and shows a very conservative behavior, but might be used in onboard obstacle avoidance or as a back up algorithm in case of the creation of a local minimum.
\end{abstract}

%%%%%%%%%%%%%%%%%%%%%%%%%%%%%%%%%%%%%%%%%%%%%%%%%%%%%%%%%%%%%%%%%%%%%%%%%%%%%%%%

%!TeX spellcheck = en_US
%!TeX encoding = UTF-8
%!TeX root = ../report.tex

\section{Introduction} \label{sec:introduction}
In classical robotics, real time path planning is often used to operate the platforms. Based on the position and goal, a path is planned along which the controller tries to move the robot. In case of a disturbance, which can be of various form (a human walking in front of a factory robot or a unmanned drone been by a wind gust) the robot has to recover. A classical controller tries to reevaluate the motion. But replanning is expensive, and therefore there are two possibilities: stop and wait for the path planning to finish, or to go back to the position before the disturbance. In either way, this results in a quirky, unnatural motion, which robots are  known for. \\
A more recent approach to tackle this problem is based on Dynamical Systems (DS), in order to evaluate the desired velocity. This allows for a dynamic adaptation of the motion and fast replanning. Moreover, the resulting movement resembles human like motion. An important aspect is, that a well chosen DS is continuously defined and ensures the completion of the task. Furthermore the robot can instantly adapt to a changing environment without performing an expensive replanning of a path.\\

DS have emerged as one of the most general and flexible ways to represent motion plans for robotic applications. In contrast to classical architectures, where a robot is usually programmed to track a given reference position trajectory as accurately as possible, in DS representations the static reference trajectory is replaced by one which unfolds as the task progresses. \\
DS based approaches to robot control offers robustness and smooth adaptation to real-time perturbations. The robot motion in a DS unfolds in time with no need to re-plan \cite{khansari2012dynamical} \\
The state variable $\xi  \in \mathbb{R}^d $ defines the the state of robotic system. Its temporal evolution $\dot \xi$ may be given by either an autonomous (time-invariant) or non-autonomous (time-varying) DS as described in:
\begin{align} \label{eq:DS_sys}
  &\dot \xi = f(\xi) & \;\; & f: \mathbb{R}^d \mapsto \mathbb{R}^d & \;\; & \text{autnonomous DS}\label{eq:DS_sys} \\
  &\dot \xi = f(t,\xi)& \;\; &  f: \mathbb{R}^+ \times \mathbb{R}^d \mapsto \mathbb{R}^d & \;\; &  \text{non-aut. DS} \label{eq:DS_sys2}
\end{align}

where $f(\cdot)$ is a continuous function. \\

Moreover, there is special class of autonomous DS are the linear DS with a single attractor at $\xi^a$ defined as:
\begin{equation}
\dot \xi = f_l(\xi) \quad \text{with:} \;\; \frac{f_l(\xi)}{\| f_l(\xi) \|}  = \frac{-(\xi - \xi^a)}{\| \xi - \xi^a\|} \label{eq:DS_lin}
\end{equation}
these play an important role in robotics. They are often used, for a task where the robot has to reach a goal position, because they use the most direct path. Examples for these are pick and place tasks, or button push tasks. They analysis of the algorithm in this paper will be focusing on this class of DS.

\section{Related Work}
Obstacle avoidance is an old problem in robotics and many approaches have been proposed. It is often differentiated between global and local methods. Global methods are referred to as path planning; different methods are used in \cite{kavraki1996probabilistic,lozano1983spatial}. While they guarantee to find a feasible solution if there exists one, most of the algorithms are computationally expensive \cite{toussaint2009robot}. As a result, they were developed and used as off-line algorithms only applicable in static environments. This limitations was overcome by online (partial) replaning \cite{ferguson2006replanning,}.

In contrast, local methods apply a \textit{deformation}, which is a real-time path adaptation through local reshaping of the planned trajectory without the use of the global path planning algorithm. The \textit{elastic band} algorithm is such a method \cite{quinlan1993elastic, brock2002elastic}, which adapts the plan through local energy minimization. While these methods allow fast remapping, it is not guaranteed that a feasible solution is found.

A solution for this problems are hybrid algorithms, which combine global path planning and local obstacle avoidance \cite{yoshida2011reactive}. They allow to switch to a global algorithm, when the local one fails to find a feasible solution. Conversely, they use the local inner loop while a global, computational complex path planning is under process. It is to notice, that the hybrid algorithms provide a closed form solution to avoid the obstacles \cite{vannoy2008real,}.

Another local method is the introduction of an artificial potential field, where each obstacle is represented by a potential function \cite{rimon1992exact}. This  method was extended to dynamic environment and was applied on a robot arm, where a collision free path for all links could be computed and followed in experiments \cite{khatib1986real}. While a collision free path can be guaranteed by defining an infinite potential at obstacle boundaries, the convergence to the global minimum is not ensured. In fact, it can be shown that a smooth vector field will have at least as many stationary points as obstacles; in the ideal case with good choice of the obstacle fields those equilibrium are saddle points, in the worst case they are local minima \cite{koditschek1990robot}.

Local extrema can be removed, while ensuring global avoidance by taking inspiration in the description of the dynamics of fluids around impenetrable obstacles. The analytical description of the harmonic functions, which are used to describe the stream lines around the obstacles, can be used to control robots. An important property of harmonic functions is, that all minima and maxima are on the boundaries of the space \cite{kim1992real}. This was used in the control of a 3DOF robot, but was limited to static objects, further, only one obstacle is considered at a time \cite{guldner1993sliding}. It was extended to moving obstacles with constant velocities (translational and rotational) \cite{feder1997real}.
%Harmonic potential based methods are powerful in  that they do not have local minima.The traditional potential function can be augmented to take the stream function of the field into account \cite{daily2008harmonic}.While this method allows the avoidance of circular obstacles, it fails to avoid combined obstacles.

More recent approaches use DS for the implementation of obstacle avoidance algorithms, e.g. \cite{ijspeert2002movement,khansari2012dynamical}. For example a biological inspired approach includes the use of DS for movement planing for the control of robots, where it allows for robust motion planing in the presence of disturbances, while having dynamic human-like motion. The model was extended with description of human obstacle avoidance,  this allows the avoidance of several point like obstacles \cite{hoffmann2009biologically}.

Modification of the original DS can be done by local rotations \cite{kronander2015incremental}. This is used to improve a learned DS (through various machine learning algorithms) to perform tasks such as writing letters. The local rotation can also be used to locally modulate a linear DS and therefore enforce a robot to follow a specific path. This procedure of local rotation of the DS has not been used for the specific task of obstacle avoidance so far.

Using harmonic functions for obstacle avoidance with DS allows to create smooth trajectories for point-robots around circular obstacles \cite{waydo2003vehicle}. While this method was originally introduced to create smooth paths for less agile robots such as airplanes, it can be extended to smooth path generation for any platform. A dynamic modulation matrix was used to adapt the DS in \cite{khansari2012realtime,}, which allowed to avoid convex static obstacles. The algorithm has been extended to moving obstacles in \cite{khansari2012dynamical}. It was successfully implemented on an arm manipulator to avoid a single, fast approaching obstacle. Local minima of the harmonic function which describes the modulated DS only appear on the obstacle's boundary. The minima are exited by switching to a different control mode until the robot is attracted to the global minimum. Concave obstacles are treated conservatively by creating a convex hull around them. Moreover, an analytic description of the obstacle is needed for this algorithm, which is often computationally expensive to find. \\
The approach could be extended to an obstacle description based on point clouds \cite{saveriano2013point,}. Besides, the adapted algorithm allowed to avoid concave obstacle in discrete space; which was tested by picking objects out of a (concave) box. A second adaptation of the algorithm to moving obstacles was achieved  \cite{saveriano2014distance}.


% with its help the DS is modulated to avoid the obstacle. The important aspect is, that the matrix has no effect far away from the obstacle, but close to the obstacle it has to remove the radial velocity. The algorithm is implemented for ellipse like obstacle, but can be extended to any convex obstacle. No solution is proposed for concave obstacles, apart from forming a convex hull. A similar modulation matrix and the use of DS has been proposed in \cite{saveriano2014distance,saveriano2014distance,} to avoid obstacles. The algorithm allows the adaptation of safety margin, reactivity and introduces the possibility to stop the modulation in the wake of an obstacle. The extension to moving obstacles ensures a collision free trajectory by ease the avoidance around the wake of the obstacle.


This report at hand starts with a brief description and analysis of obstacle avoidance with a dynamic modulation matrix as proposed by \cite{khansari2012dynamical} in Sec.~\ref{sec:DMM}. It continues with the description and analysis of the algorithm developed by the authors of this report with obstacle avoidance inspired by fluid dynamics, as presented in Sec.~\ref{sec:IFD}. %Furthermore, obstacle avoidance by local rotation of space is mentioned in Sec.~\ref{sec:LRS} but further analyzed in the appendix. A more extensive qualitative and quantitative comparison of the two main algorithms is performed in  Sec.~\ref{sec:comparison} with the help of further simulations.
This is followed by discussion (Sec.~\ref{sec:discussion}) and the conclusion.
% \todo{make more}


%\cite{saveriano2013point}


%%% Local Variables:
%%% mode: latex
%%% TeX-master: "../main"
%%% End:


\input{sec/OA_dynamicModulationmatrix}

\section{Obstacle Avoidance Inspired by Fluid Dynamics (IFD)} \label{sec:IFD}
An important technique in fluid dynamics for describing flow around obstacles in two dimensions, is the use of conformal mapping \cite{milne1996theoretical,feder1997real}. One of the most common mapping is the Joukowski transformation, which allows to describe the transformation from a unit circle at the origin to any arbitrary positioned ellipse. We want to do a similar procedure by taking the algorithm described in Sec.~\ref{sec:DMM}, apply it to a unit circle and then transform the coordinate system such that the unit circle becomes an ellipse.

The algorithm in this section is described for ellipses in two dimensions, where the boundary is given as:
\begin{equation}
\mathcal{X}^b \subset \mathbb{R}^2 = \{ \xi \in \mathbb{R}^2 : \frac{\tilde \xi_1^2}{a_1^2} + \frac{\tilde \xi_2^2}{a_2^2} = 1 \} \label{eq:ellipse_2d}
\end{equation}
with the position $\tilde xi$ relative to the center $\xi^o$ described in the coordinate system aligned with the main axis of the ellipse $a_1$, $a_2$.\\
Furthermore, it is expected that the algorithm is  extendable to general convex obstacles in $d$ dimensions. % r the description of the velocity, Cartesian and cylindric coordinates are used, the conversion is given as follows:

\subsection{Algorithm}
The modulation of the DS with a convex obstacle has been described in \ref{eq:modMatrix}. It was furthermore observed, that this algorithm did perform well with a circle but has limitations with more complex obstacles. We will therefore base the new algorithm on this local modulation of a unit circle. The level function $\Gamma(\tilde \xi)$ used to caclulate the eigenvalues in (\ref{eq:eigValues}) is given for the unit circle as:
\begin{equation}
\Gamma(\tilde \xi) = \| \tilde \xi \|^2 \label{eq:levelCircle}
\end{equation}
The coordinate system, is then scaled with the value of the main axis of the ellipse $a_1$ and $a_2$. The level function for this new ellipse obstacle is now given as:
\begin{equation}
\Gamma(\tilde \xi) = \frac{\tilde \xi_1^2}{a_1^2} +  \frac{\tilde \xi_2^2}{a_2^2} \label{eq:levelEllipse}
\end{equation}

\noindent \textbf{Theorem 1}
\textit{Consider a linear stretching of space such that a unit spherical obstacle in $\mathbb{R}^d$ with center ${}_c \xi^o$ and radius ${}_cr^o = 1$, with surface $\mathcal{X}^b \subset \mathbb{R}^2  = \{ \xi \in \mathbb{R}^2: \| \xi - {}_c\xi^o \| = 1\}$, is transformed into an ellipse defined in (\ref{eq:ellipse_2d}). Any tangent  on the unit circle ${}_c \vec{e}_i$ which is transformed in such a way, will be a tangent on the ellipse (${}_e \vec{e}_i$). This is not the case for normal vectors on the circle surface ${}_c \vec{n}$,  which are not normal to the ellipse after transformation.}\\
\textbf{Proof:} See Appendix~\ref{sec:proof1} \\
${}$ \hfill $\blacksquare$\\

%Any transformation $\mathcal{X}^o \rightarrow \mathcal{X}^s : \mathcal{X}^d \leadsto \mathcal{X}^d$ which keeps a closed hyper-surface intact and ensures a smooth first order Jacobian will transform a tangent of the later hyper-surface to a tangent in the new space.
Fig.~\ref{fig:comparison_behaviour} illustrates the tangent (green) and the normal (red) before and after transformation for both the IFD, additionally it shows it for the DMM around the same ellipse. \\
At this point, we introduce the unit circle reference frame (CRF) which is indicated by indice ${}_c (\cdot)$ and the ellipse reference frame (ERF) with the indice ${}_e(\cdot)$. \\
It can be observed, that the tangent on the circle ${}_c \vec e_1(\tilde \xi)$ is a tangent of the ellipse after the transformation ${}_e \vec e_1(\tilde \xi)$. On the other hand, the normal on the circle ${}_c \vec n (\tilde \xi)$ is not a normal on the ellipse after the transformation.This transformed normal on the unit circle will further be referred to as \textit{pseudo normal} $\hat{\vec n} (\tilde \xi)$  in the ellipse space and is given as:
\begin{equation}
\hat{\vec n }(\tilde \xi) = {}_e\vec n_{c} (\tilde \xi) =   \frac{1}{\sqrt{ {}_e\tilde \xi_1^2 + {}_e \tilde \xi_2^2}}
  \begin{bmatrix}
     {}_e\tilde \xi_1 \\
     {}_e\tilde \xi_2
   \end{bmatrix}
\end{equation}

The modulation matrix in (\ref{eq:modMatrix}) needs to be adapted to for the IFD. And we get a new modulation matrix $\hat M$:
\begin{equation}
\hat M = \hat E(\tilde \xi)  D(\tilde \xi) \hat E^{-1}(\tilde \xi)\label{eq:modMatrix_IFD}
\end{equation}IFw
The normal in $E(\xi)$ is replaced with the pseudo normal in $\hat E(\tilde)$:
\begin{equation}
  \hat E(\tilde \xi) = \begin{bmatrix} \hat{\vec n}(\tilde \xi) & \vec e_1(\tilde \xi) \end{bmatrix}
\end{equation}
The basis vectors for the IFD which construct $\hat E$ are not necessarily orthogonal, but only linearly independent.

The modulation of the DS follows analogous to (\ref{eq:modulatedDS}) as:
\begin{equation}
  \dot \xi = \hat{M}(\xi) f(\xi) \label{eq:modulatedDS_IFD}
\end{equation}

\\ \noindent\textbf{Theorem 2}
\textit{Consider a 2-dimensional elliptic obstacle in $\mathbb{R}^2$ as given in (\ref{eq:ellipse_2d}). Any motion $\{\xi\}_t$, $t = 0 .. \infty$ that starts outside the obstacle, i.e. $ \frac{(\{\xi_1\}_0 - \xi_1^o)^2}{a_1^2} + \frac{(\{\xi_2\}_0 - \xi_2^o)^2}{a_2^2} > 1  $ and evolves according to (\ref{eq:modulatedDS_IFD}) never penetrates into the obstacle, i.e.$ \frac{(\{\xi_1\}_t - \xi_1^o)^2}{a_1^2} + \frac{(\{\xi_2\}_t - \xi_2^o)^2}{a_2^2} > 1 $.} \\
\textbf{Proof:} See Appendix~\ref{sec:proof2} \\
${}$ \hfill $\blacksquare$


While impenetrability of the obstacle could already be ensured by the DMM, a limitation of the later was the observation of local minima on the surface of ellipse shaped obstacles. The IFD was designed to overcome this limitation, where it does no create local minima on the ellipse surface. The goal is to remove the local extrema, but it could be shown, that a smooth vector field in a sphere world must have at least as many extrema as obstacles \cite{koditschek1990robot}. In the optimal case this extrema can be turned into a saddle point, where points on one line converges to it. With our current modulation matrix $\hat M(\tilde \xi)$ and in a linear DS $f_l$, all the points on this saddle point trajectory have an initial DS which points directly in direction of the pseudo normal $\hat{\vec n}$. As a result of this, the modulation with $\hat M(\xi)$ only causes a change in magnitude of the velocity, but does not turn the velocity to the right or left. The saddle point trajectory is given by:
% \begin{equation}
%   \begin{split}
%   \mathcal{X}^s \subset \mathbb{R}^2 = \{\xi \in \mathbb{R}^2: &\frac{1}{\|f_l(\xi)\|} f_l(\xi) = - \hat{\vec n}(\tilde \xi),\\
%   & \|\xi - \xi^a\| > \| \xi - \xi^o\| \} \label{eq:centerLine}
%   \end{split}
% \end{equation}.\\

\begin{equation}
  \begin{split}
  \mathcal{X}^s \subset \mathbb{R}^2 = \{\xi \in \mathbb{R}^2: \; f_l(\xi) \parallel \hat{\vec n}(\tilde \xi), \} \label{eq:centerLine}
  \end{split}
\end{equation}.\\

\noindent \textbf{Theorem 3}
\textit{Consider a time invariant, linear DS $f_l(\xi)$ with a single attractor placed at $\xi^a$ given by (\ref{eq:DS_lin}). The DS is modulated according to (\ref{eq:modulatedDS_IFD}). Any motion $\{\xi\}_t$ that starts outside the obstacle and is not on the saddle point trajectory as given in (\ref{eq:centerLine}) i.e. $\{ \{\xi\}_0 \in \mathbb{R}^d \setminus \mathcal{X}^s \; : \; \Gamma(\{\xi\}_0) \geq 0\}$ will move around the obstacle and converge to the attractor $\xi^a$.} \\
\textbf{Proof:} See Appendix~\ref{sec:proof3} \\
${}$ \hfill $\blacksquare$ \\
Fig.~\ref{fig:comparison_behaviour} shows very well, at every point in space the DS is modulated in the right direction, such that it doe not converge to a local minimum as could be observed with the DMM.
\begin{figure*}[tb]\centering
\begin{subfigure}{.215\textwidth} %
\centering
\includegraphics[width=\textwidth]{fig/avoidingEllipse_DMM.eps}
\caption{DMM in ERF}
\label{fig:avoidingEllipse_DMM}
\end{subfigure} %
\begin{subfigure}{.215\textwidth} %
\centering
\includegraphics[width=\textwidth]{fig/avoidingEllipse_IFD_ERF.eps}
\caption{IFD in ERF}
\label{fig:avoidingEllipse_IFD_ERF}
\end{subfigure}
\begin{subfigure}{.285\textwidth} %
\centering
\includegraphics[width=\textwidth]{fig/avoidingEllipse_IFD_CRF.eps}
\caption{IFD in CRF}
\label{fig:avoidingEllipse_IFD_CRF}
\end{subfigure}
\begin{subfigure}{.24\textwidth} %
\centering
\includegraphics[width=\textwidth]{fig//staticEllipse_IFD_time0.eps}
\caption{Streamlines of DS with IFD around ellipse.}
\label{fig:avoidingEllipse_IFD}
\end{subfigure}
\caption{Fig. (a) \& (b) show the comparison between the two alogirthms, where the tangent (green) stays the same, but the normal (red) changes. Furthermore, initial linear DS (blue arrows) is modulated differently to the final dynamics (blue arrows), as a result of the basis vectors. The pseudo normal (green) and tangent (red) in (a) is evaluated by stretching the space and the corresponding actual normal and tangent on the uniit circle in (c). It can be seen in (d) that this new modulation rotates the DS away from the centerline and around the ellipse.}
\label{fig:comparison_behaviour}
\end{figure*}



\vspace{1ex} \noindent
\textbf{No Change of the Tangential Velocity:}
Moreover, the sufficient condition for convergence for the eigenvalues is:  $\lambda_n(\tilde \xi) < \lambda_1 (\tilde \xi)$. This allows more flexibility in their choice than what is proposed by \cite{khansari2012dynamical,}. Under the intention to modify the DS as little as possible while pensuring impenetrability, the eigenvalue corresponding to the tangent in (\ref{eq:eigValues}) is set to one ($\lambda_1 = 1$). The initial value is an result of the incompressibility condition of the fluid,  which needs to accelerate the DS in tangential direction to get the same flow volume through a smaller area. This is not necessary for our application of the obstacle avoidance in robotics.\footnote{Keeping the incompressibility condition might be of interest, when controlling a swarm of robots around an obstacle to avoid them from colliding.}


\section{Comparison}
A good performance in a DS with four static obstacles is achieved with the IFD, where most trajectories converge to the global minimum (Fig.~\ref{fig:singleAttractor_severalObstacles_time}). On the other hand, the elongated ellipses poses a problems for the DMM as a local minimum is formed on the surface of the largest one (southwest), but also for the smaller one the algorithm does not seem to perform well. Moreover, several ellipses and narrow pathways are handles very well by both algorithms.
\begin{figure}[tb]\centering
\begin{subfigure}{.48\columnwidth} %
\centering
\includegraphics[width=\textwidth]{fig/singleAttractor_severalObstacles_DMM_time0.eps}
\caption{DMM}
\label{fig:singleAttractor_severalObstacles_DMM_time0}
\end{subfigure}%
\begin{subfigure}{.48\columnwidth} %
\centering
\includegraphics[width=\textwidth]{fig/singleAttractor_severalObstacles_IFD_time0.eps}
\caption{IFD}
\label{fig:singleAttractor_severalObstacles_IFD_time0}
\end{subfigure}
\caption{Obstacle avoidance of four static obstacle placed in a DS with a single attractor.}
\label{fig:singleAttractor_severalObstacles_time}
\end{figure}

The avoidance of an arbitrary placed circular obstacle is easily avoided with both algorithms (Fig.\ref{fig:static_circle}). In fact, no real difference of the behavior can be observed and this was expected, because the IFD uses the DMM on a circular obstacle as a starting evaluation. Only for more complex shapes, a difference and also the advantage of the IFD can be observed.
\begin{figure}[tb]\centering
\begin{subfigure}{.48\columnwidth} %
\centering
\includegraphics[width=\textwidth]{fig/oneStaticCirc_DMM_time0.eps}
\caption{DMM}
\label{fig:oneStaticCirc_DMM}
\end{subfigure}\,\, %
\begin{subfigure}{.48\columnwidth} %
\centering
\includegraphics[width=\textwidth]{fig/oneStaticCirc_IFD_time0.eps}
\caption{IFD}
\label{fig:oneStaticCirc_IFD}
\end{subfigure}\,\, %
\caption{Obstacle avoidance of a static circle placed in a linear DS.}
\label{fig:static_circle}
\end{figure}



% \vspace{1ex} \noindent
% \textbf{Emergency Exiting:}
% While in continuous time and space the robot will never enter the surface, this is not ensured in with discrete integration. If during a simulation, the robot enters the safety margin of an obstacle, the algorithm switches to the emergency mode. In this mode the robot moves away from the center as follows:
% \begin{equation}
% \begin{bmatrix}
% \dot{\bar{\xi}}_r \\
% \dot{\bar{\xi}}_{\theta}
% \end{bmatrix}
% =
% \begin{bmatrix}
% v_{exit} \\
% 0
% \end{bmatrix}
% \qquad \forall \; \xi_r < R
% \end{equation}
% with $v_{exit}$ being a constant exiting velocity.

% \subsubsection{Ellipse Obstacle}
% The extension to ellipse obstacles can be done by applying a transformation to the whole space, that in the new space the object is a circle. The position and velocity are scaled in the following way:
% \begin{equation}
% {\xi}_{circ} = A^{-1} \cdot {\xi}_{ellips} \hspace{1cm} \dot{\xi}_{circ} = A^{-1}  \cdot \dot{\xi}_{ellips}
% \end{equation}
% with the linear matrix $A$ as a function of the ellipse axis $a$ and $b$
% \begin{equation}
%   A =
%   \begin{bmatrix}
%     a & 0 \\
%     0 & b
%   \end{bmatrix}
% \end{equation}
% The obstacle in the new space is a unit circle and the modulation seen in Subsection~\ref{sec:basicCircle} can be applied. \\
% The modulated DS $\dot{\bar{\xi}}_{circ}$ has then to be transformed back to the original space:
% \begin{equation}
% \dot{\bar{\xi}}_{ellips} = A  \cdot \dot{\bar{\xi}}_{circ}
% \end{equation}

% \subsubsection{Rotated Cylinder}
% If a cylinder is not aligned with the coordinated axis, the space is rotated around the origin:
% \begin{equation}
% \xi_{alligned} = R(\phi) \cdot \xi_{rot} \hspace{1cm} \dot{\xi}_{alligned} = R(\phi)  \cdot \dot{\xi}_{rot}
% \end{equation}
% with the rotation matrix $R(\phi)$ as a function of the ellipse's orientation $\phi$.

% The modulated DS is rotated back to the original space:
% \begin{equation}
% \dot{\bar{\xi}}_{rot} = R^{T}  \cdot \dot{\bar{\xi}}_{alligned}
% \end{equation}

% \subsubsection{General Position}
% A cylinder at general position can be easily observed by a translation of the coordinate system:
% \begin{equation}
% \xi_{centered} = \xi - \xi_0
% \end{equation}
% with the $\xi_0$ being the vector to the center of the ellipse. \\
% The velocity is not affected by a coordinate transformation, therefore no reverse operation is required.

% \vspace{1ex} \noindent

% %%\textbf{Impenetrability}
%Any coordinate transformation $\mathcal{E} \to \mathcal{C}:  \mathbb{R}^d \to \mathbb{R}^d$ which transforms a convex region in $\mathcal{X}_1^d \subset \mathcal{E}$ to a convex region in $\mathcal{X}_2^d \subset \mathcal{C}$, ensures that any vector starting and ending in the convex region $\dot \xi_{\mathcal{E}} \in \mathcal{X}_1$ will be contained in the convex region $\dot \xi_{\mathcal{C}} \in \mathcal{X}_2$, vise versa any vector outside the starting and ending outside will start and end outside this region.

%The linear transformation with the stretching matrix $A$ and rotation matrix $R$, as well as the translation with $\xi_0$ keep the convex regions intact. As the velocity vector starts and stays outside the concave circle and will therefore stay that way for the transformation back to the ellipse space, while impenetrability is  given. \hfill $\blacksquare$ \vspace{1em}
\noindent
\textbf{Impenetrability}
The surface of the ellipse can be described as:
\begin{equation}
 \frac {{}_{e}\xi_x ^ 2}{a^2} + \frac{{}_e\xi_y^2}{b^2} = 1
\end{equation}
The subscript ${}_e (\cdot)$ declares that a variable is given in the ellipse reference frame (ERF).
With the Jacobian, a normal ${}_e \vec n_e$ and unit normal ${}_e \vec n_{e,0}$ vector on the ellipse can be found:
\begin{equation}
  {}_e \vec{n}_e =
  \begin{bmatrix}
    \frac{2 \cdot {}_e \xi_x}{a^2} \\
    \frac{2 \cdot {}_e \xi_y}{b^2}
  \end{bmatrix}
  \qquad
  {}_e \vec{n}_{e,0} =
  \frac{1}{\sqrt{b^4 \cdot {}_e\xi_x^2 + a^4 \cdot{}_e \xi_y^2}}
  \begin{bmatrix}
    b^2 \cdot {}_e\xi_x \\
    a^2 \cdot {}_e\xi_y
  \end{bmatrix}
\end{equation}

The unit tangent on the ellipse ${}_e\vec t_{e,0}$ is perpendicular to the normal, and can be written as:
\begin{equation}
  {}_e \vec{t}_{e,0} =
  \frac{1}{\sqrt{b^4 \cdot {}_e\xi_x^2 + a^4 \cdot{}_e \xi_y^2}}
  \begin{bmatrix}
    a^2 \cdot {}_e\xi_y \\
    - b^2 \cdot {}_e\xi_x
  \end{bmatrix}
\end{equation}

Before the modulation is applied, the whole coordinate system is transformed, such that the ellipse lies on a unit circle, which is referred as circle reference frame (CRF) with the subscript ${}_c(\cdot)$:
\begin{equation}
  {}_c \xi = A^{-1} {}_e \xi \quad \rightarrow \quad {}_e \xi_x = a \cdot {}_c\xi_x \, , \;\; {}_e \xi_y = b \cdot {}_c\xi_y \label{eq:ellipseCircleTrafo}
\end{equation}
For a unit circle, the two axis have the same value, and are given by $r = a = b = 1$.
\begin{equation}
 \xi_x ^ 2 + \xi_y^2 = 1
\end{equation}
The normal ${}_c \vec n_ {c}$ and the unit normal ${}_c \vec n _{c,0}$ of this circle in CRF can be evaluated similarly
\begin{equation}
  {}_c \vec{n}_c =
  \begin{bmatrix}
    {2 \cdot {}_c \xi_x} \\
    {2 \cdot {}_c \xi_y}
  \end{bmatrix}
  \qquad
  {}_c \vec{n}_{c,0} =
  \frac{1}{\sqrt{ {}_c\xi_x^2 + {}_c \xi_y^2}}
  \begin{bmatrix}
     {}_c\xi_x \\
     {}_c\xi_y
  \end{bmatrix}
\end{equation}

and the tangent
\begin{equation}
  {}_c \vec{t}_{c,0} =
  \frac{1}{\sqrt{ {}_c\xi_x^2 + {}_c \xi_y^2}}
  \begin{bmatrix}
     {}_c\xi_y \\
     - {}_c\xi_x
  \end{bmatrix}
\end{equation}

On the obstacle surface, the velocity needs to be solely in tangential direction. While this was shown for the circle, the proof has to be extended to the ellipse. For this, the tangent on the unit circle needs to be equal to the tangent of the ellipse, too. To evaluate this, the tangent on the unit circle in CRF is transformed to the ERF:
\begin{equation}
  {}_e \vec{t}_{c} = A \cdot {}_c \vec{t}_{c,0}
   \qquad
   {}_e \vec{t}_{c,0} = \frac{1}{\sqrt{a^2 \cdot {}_c\xi_y^2 + b^2 {}_c\xi_x^2}}
  \begin{bmatrix}
       a \cdot {}_c\xi_y \\
     - b \cdot {}_c\xi_x
   \end{bmatrix}
\end{equation}
Furthermore, the coordinate transformation can be applied (Eq.~\ref{eq:ellipseCircleTrafo}) and leads to:
\begin{equation}
  {}_e \vec{t}_{c,0} =
  \frac{1}{\sqrt{b^4 \cdot {}_e\xi_x^2 + a^4 \cdot{}_e \xi_y^2}}
  \begin{bmatrix}
    a^2 \cdot {}_e\xi_y \\
    - b^2 \cdot {}_e\xi_x
  \end{bmatrix}
   =
    {}_e \vec{t}_{e,0}
\end{equation}
which is equal to the tangent on the ellipse in ellipse space. Projecting the velocity solely on the tangent in ellipse space is ensured and therefore impenetrability is guaranteed. \hfill $\blacksquare$ \\

It is important to notice, that the normal on the circle transformed to the ERF ${}_e \vec n_{c,0} $ is not normal to the ellipse:
\begin{equation}
  {}_e\vec n_{c,0}
  =
  \frac{1}{a b\sqrt{ {}_e\xi_x^2 + {}_e \xi_y^2}}
  \begin{bmatrix}
     ab \cdot{}_e\xi_x \\
     ab \cdot {}_e\xi_y
   \end{bmatrix}
  \neq
  {}_e\vec n_{e,0}
\end{equation}
As a result of this, the normal and tangent of the CRF are not orthogonal in ERF, but linearly independent. Zero velocity in direction of the normal on the ellipse can still be ensured, because the normal on the ellipse ${}_e \vec n_{e,0}$ is a linear combination of the normal on the circle in ERF (${}_e \vec n_{c,0}$, ${}_e \vec t_{e,0}$). As there is no velocity in direction of ${}_e \vec n_{c,0}$ it is a trivial normal combination resulting in ${}_e \vec n_{e,0} = 0$.

%%% Local Variables:
%%% mode: latex
%%% TeX-master: "../main"
%%% End:

% \\

% \subsubsection{Moving Object}
% The movement of an obstacle is taken into account by changing the frame of reference to the velocity reference frame (ORF). This affects the DS in following way:
% \begin{equation}
% \dot{\tilde{\xi}}=  \dot{\xi} + \dot{\xi}^{obs}
% \end{equation}
% The position vector is not affected by the movement. \\

% The modulated DS is transformed back to the inertial reference frame (IRF):
% \begin{equation}
% \dot{\bar{\xi}} =  \dot{\tilde{\bar{\xi}}} - \dot{\xi}^{obs}
% \end{equation}

% \vspace{1ex} \noindent
% \textbf{No Wake Effect Behind Obstacle}
% As a result of the incompressible liquid asssumption, there is  a wake effect behind an obstacle. The fluid modulated to move tangential to the surface, while it could move away from it without collision.\footnote{This is also referred as \textif{tail effect} in other algorithms \cite{khansari2012dynamical}}  In the extreme case, a robot on the obstacles boundary might be pulled along with the moving obstacle. \\
% The wake effect is reduced, by modifying the velocity in normal direction only if it is moving towards the obstacle. This is done by modifying the first eigenvalue:
% \begin{equation}
%   \lambda_1(\tilde \xi) =
%   \begin{cases}
%     f(\tilde \xi) \quad &  \text{if} \; \; n(\tilde \xi)^T \cdot \dot \xi < 0 \\
%     1 \quad & \text{otherwise} %n(\tilde \xi)^T \cdot \dot \xi \geq 0 \\
%   \end{cases}
% \end{equation}

% \subsubsection{Rotating Ellipse} \label{sec:rotatingEllipse}
% A rotation of an object is treated in a similar way as a moving object, where the original DS is represented in the rotating frame of reference. The instantaneous velocity of the object is considered at the robot's position:
% \begin{equation}
% \dot{\xi}_{ellipse}^{obs}(\xi) =
% \begin{bmatrix}
% d \cdot \dot \phi \\
% 0
% \end{bmatrix}
% \end{equation}
% with the $d$ the distance of the robot to the center and $\dot \phi$ the angular velocity of the object. This latter is handled in the same way as a velocity of an object, but its value dependends on the position. Moreover, the angular velocity needs to be evaluated before the linear stretching of space.

% \vspace{1ex} \noindent
% \textbf{No Turbulence Outside Largest Axis Radius}
% A rotating ellipse in an incompressible fluid causes turbulence even beyond the largest axis. This is an undesired effect as it does not help the collision avoidance. \\
% The algorithm is therefore adapted to consider the angular rate of an object if and only if it's within the radius of the largest axis of the object, but is not considered far away. To create a smooth DS, a transition region is chosen where the two options are interpolated (Fig.~\ref{fig:trans_rotEllipse}).

% \vspace{1ex} \noindent
% \textbf{Impenetrability}
% The condition of staying in the ORF which ensures nonpenetrability (sec.~\ref{sec:basicCircle}) is not met in favor of minimizing the wake effect and turbulence. Nonpenetrability can still be ensured, because the ORF is always chosen when the DS represented in the ORF is moving towards the object. \hfill $\blacksquare$
% \vspace{1em}

% \subsubsection{Several Objects}
% As the superposition of a harmonic function is again a harmonic function, the  DS can be modulated on each obstacle separately and superposed afterwards.  This results in a function without local minima away from the boundaries, but does not satisfy the impenetrability condition. To resolve this, in regions close to an obstacle all modulated DS, apart from the one of the closest obstacle, are modulated and superposed. The modulation of the last obstacle is then applied on this superposed DS. Between the regions far away and close, there is an interpolation region to generate a smooth transition of the DS (Fig.~\ref{fig:trans_severalEllipse}). The DS for several static obstacles is therefore given as:
% \begin{equation}
%   \dot{\bar \xi} =
%   \begin{cases}
%     f_{mod,k}(\frac{1}{N-1} \sum_{n \neq k} \dot{\bar \xi}_n) & \text{close to obstacle k}\\
%     \frac{1}{N} \sum_n \dot{\bar \xi}_n \quad & \text{far away} \\
%     \text{transition / interpolation} \quad & \text{otherwise}
%   \end{cases}
% \end{equation}
% with $N$ the number of obstacles and $F_{mod,k}$ the IRF procedure applied on the obstacle $k$ as described above.

% \vspace{1ex} \noindent
% \textbf{Impenetrability}
% Impenetrability can be ensured only when there is a finite space between two obstacles. Close to obstacles, the closest obstacle is considered on a general DS and the previous proof of Sec.~\ref{sec:basicCircle} can be applied.  \hfill $\blacksquare$
% \vspace{1em}

% \begin{figure}[tb]\centering
% \begin{subfigure}{.48\columnwidth} %
% \centering
% \includegraphics[width=\textwidth]{fig/transition_rotatingEllipse.eps}
% \caption{Rotating Ellipse}
% \label{fig:trans_rotEllipse}
% \end{subfigure}
% \begin{subfigure}{.48\columnwidth} %
% \centering
% \includegraphics[width=\textwidth]{fig/transition_severalEllipse.eps}
% \caption{Several ellipses.}
% \label{fig:trans_severalEllipse}
% \end{subfigure}
% \caption{The transition for a rotating ellipse (\ref{fig:trans_rotEllipse}) where the rotation of the obstacle is only fully considered in the red region, not considered in the white region, as well as a linear interpolation region to get a smooth transition (orange). \\
% When considering multiple obstacles (\ref{fig:trans_severalEllipse}) all harmonic functions of the obstacles are superposed far away from the obstacle (white region), in the region around the obstacles the closest one is privileged by applying it on the combined modulated DS of the other (red) and there is a linear transition region (orange).}
% \label{fig:transBoundary}
% \end{figure}
% This algorithm is limited to static obstacles and needs to be extended to moving obstacles.

% \subsubsection{Concave Objects}
% The algorithm extension for concave objects has not been implemented yet. A suggestion by the author is to describe any concave object as a combination of ellipses. The surface of these ellipses in the concave region could be represented as fluid dynamic equivalent of a continuous line distribution of vortexes. This will add a tangential velocity in direction away from the point of intersection. This tangential velocity is inverse proportional to the distance to the surface of the object. Therefore, it has an infinite value on the surface. The direction of the velocity vector on the surface is therefore tangential to the surface on the obstacle boundary. The DS will not penetrate the obstacle.  \\ %\vspace{1ex}




% \begin{figure}[tb]\centering
% \centering
% \begin{subfigure}{.49\columnwidth} %
% \centering
% \includegraphics[width=\textwidth]{fig/avoidingEllipse_IFD_ERF.eps}
% \caption{IFD in ERF}
% \label{fig:avoidingEllipse_IFD_ERF}
% \end{subfigure}
% \begin{subfigure}{.49\columnwidth} %
% \centering
% \includegraphics[width=\textwidth]{fig/avoidingEllipse_DMM.eps}
% \caption{DMM in ERF}
% \label{fig:avoidingEllipse_DMM}
% \end{subfigure} %
% \caption{The original DS (gray arrows) and the modulated one (blue arrows) on an ellipse obstacle (green patch). The red (normal) and green line (tangent) are evaluated for the IFD in CRF and the DMM in ERF. For the IFD in ERF they are corresponding values after the transformation CRF; while the green line are the tangent on the ellipse, the red line is not the normal anymore.}
% \label{fig:avoidingEllipse_compare}
% \end{figure}

%\begin{figure*}[tb]\centering
% \begin{subfigure}{.228\textwidth} %
% \centering
% \includegraphics[width=\textwidth]{fig/avoidingEllipse_IFD_ERF_rotTrans.eps}
% \caption{Initial}
% \label{fig:avoidingEllipse_DMM}}
% \end{subfigure} %
% \begin{subfigure}{.228\textwidth} %
% \centering
% \includegraphics[width=\textwidth]{fig/avoidingEllipse_IFD_ERF_rotated.eps}
% \caption{Centered}
% \label{fig:avoidingEllipse_DMM}}
% \end{subfigure} %
% \begin{subfigure}{.228\textwidth} %
% \centering
% \includegraphics[width=\textwidth]{fig/avoidingEllipse_IFD_noEig.eps}
% \caption{Aligned (ERF)}
% \label{fig:avoidingEllipse_IFD_ERF}
% \end{subfigure}
  % \begin{subfigure}{.228\textwidth} %
% \centering
% \includegraphics[width=\textwidth]{fig/avoidingEllipse_IFD_noEig.eps}
% \caption{Aligned (ERF)}
% \label{fig:avoidingEllipse_IFD_ERF}
% \end{subfigure}
%\end{figure*}[tb]\centering

% \subsubsection{Pseudo Code}
% A detailed description of the implementation is given in Algorithm~\ref{code:IRF} with a corresponding visualization in Fig.~\ref{fig:avoidingEllipse_IFD_CRF}.
% \begin{algorithm}[t]
%   \caption{IRF algorithm}\label{code:IRF}
%   \begin{algorithmic}[1]
%     \State Sort obstacles with descending distance

%     \For{K obstacles}
%     %\If{DS towards obstacle in VRF}
%     %\State DS in VRF
%     %\EndIf
%     \State Move to ORF
%     \State Recenter IRF around obstacle center $\xi_0$
%     % \If{DS towards obstacle in RRF}
%     % \State DS in RRF
%     % \EndIf

%     \State Rotate RF to align with ellipse axis

%     \State Scale RF to transform ellipse to unit circle

%     \State Apply modulation $M_{fluid}$

%     \State Reverse scale and rotate modulated DS

%     \State Transform modulated DS from ORF to IRF

%     \EndFor
%   \end{algorithmic}
% \end{algorithm}

% % \subsection{Results}
% The avoidance of an arbitrarily positioned circular obstacle is successfully mastered by the IRF algorithm (Fig.~\ref{fig:oneStaticCirc_IFD}). The obstacle trajectory avoid the obstacle on both sides and it resembles natural flow.
% %\subsubsection{Static Limit Cycle}
% The IRF can avoid several obstacles when being placed in a nonlinear DS with a limit cycle  (Fig.~\ref{fig:limitCicle_twoObst_IFD_time0}). The ellipse form of the initial limit cycle can still be identified far away from the obstacles and the trajectories move close to the obstacle boundaries when it is close to them.

% \begin{figure}[tb]\centering
% \begin{subfigure}{.48\columnwidth} %
% \centering
% \includegraphics[width=\textwidth]{fig/oneStaticCirc_DMM_time0.eps}
% \caption{Avoiding a circle in an original DS with a single attractor.}
% \label{fig:oneStaticCirc_DNN}
% \end{subfigure}\,\, %
% \begin{subfigure}{.48\columnwidth} %
% \centering
% \includegraphics[width=\textwidth]{fig/oneStaticCirc_IFD_time0.eps}
% \caption{Avoiding a circle in an original DS with a single attractor.}
% \label{fig:oneStaticCirc_IFD}
% \end{subfigure}\,\, %
% % \begin{subfigure}{.48\columnwidth} %
% % \centering
% % \includegraphics[width=\textwidth]{fig/limitCicle_twoObst_IFD_time0.eps}
% % \caption{Avoiding two ellipses in an original DS with a limit cycle.}
% % \label{fig:limitCicle_twoObst_IFD_time0}
% % \end{subfigure}
% \caption{Using IFD to avoid static obstacles in different DS.}
% \label{fig:static_IFD}
% \end{figure}

% \subsubsection{Two Moving Obstacles}
% All trajectories safely avoid the moving and rotating obstacles with IFD (Fig.~\ref{fig:twoMovingRotatingObs_IFD}). Additionally, the vector field seems to split evenly to go in front of or behind the obstacles, but in front of the obstacle most trajectories avoid the obstacle with at a larger distance. Furthermore, it seems that the trajectories are only influenced by the rotation of the obstacle, when they are within the radius of the largest axis. Additionally, there is no wake behind the obstacles. The minimum of the DS doesn't align with the real attractor (star).
% \begin{figure}[tb]\centering
% \begin{subfigure}{.48\columnwidth} %
% \centering
% \includegraphics[width=\textwidth]{fig/twoMovingRotatingObs_IFD_time0.eps}
% \caption{The modulated DS at initial conditions.}
% \label{fig:twoMovingRotatingObs_IFD_time0}
% \end{subfigure}\,\, %
% \begin{subfigure}{.48\columnwidth} %
% \centering
% \includegraphics[width=\textwidth]{fig/twoMovingRotatingObs_IFD_time2.eps}
% \caption{The modulated DS after 2 seconds.}
% \label{fig:twoMovingRotatingObs_IFD_time2}
% \end{subfigure}
% \caption{IFD at different time steps of obstacles with linear velocity (purple arrow) and angular speed (red arrow). }
% \label{fig:twoMovingRotatingObs_IFD}
%\end{figure}



% \subsection{Limitations}
% Moving and rotating can disturb an object, even if it is placed at the point of attraction.\\
% This is the result of the algorithm, with which the DS is modulated in the object frame. The transformed DS is not zero anymore, and the inertial modulated system won't be either. This in stabilized the point of attracted. It to consider, whether this is wanted or whether the object should stay rock still at the point of attraction.



%%% Local Variables:
%%% mode: latex
%%% TeX-master: "../main"
%%% End:


% \section{Obstacle Avoidance through Local Rotation of Space (LRS)} \label{sec:LRS}
% This algorithm is based on a modulation of the space by a loacal rotation as described by Kronander \cite{kronander2015incremental}. Where the modulation matrix is composed of a rotational part $R$ and a proportional part $(1+ \kappa)$ as follows:
% \begin{equation}
% M ( \xi ) = (1 + \kappa (\xi) ) \cdot R(\xi)
% \end{equation}
%  $\kappa(\xi)$ describes the scaling of the speed and $R(\xi)$ is the local rotation matrix.

% \subsection{Limitations}
% The rotation around an obstacle is an intuitive concept: one turns to avoid a pole which is in the way or steer a car to avoid a road bump. Analysis of this turning behavior yields, that it is linked to many factors. The proposed algorithm therefore resembles the description of the observed behavior of humans to avoid obstacles in \cite{hoffmann2009biologically}. The magnitude of the rotation is related to the distance to the obstacle. Additionally, the desired DS direction is based on position relative to the obstacle. Furthermore, the rotation of space needs to turn in both directions to achieve a split of the DS around the obstacle, as it would be observed for a large crowd which goes around both sides of a pole. While formula the local rotation is simple, the optimal choice of the parameters ($R$ and $\kappa$) is hard to define and largely position dependent on all the above factors.

% Moreover, the final algorithm can not be written in close form but includes many parameters and conditions, which makes it hard to tune. The description and simulation result are discussed in detail in the Appendix. It is necessary to adapt and tune the algorithm before it can be implemented to reliably avoid obstacles. With variables used in this report, the algorithm shows conservative behavior, which could be useful to get out of local minima if the other algorithm (IRF /DMM) fail to avoid them. On top of that, it only takes information which can be observed by sensors placed on the robot. This allows online avoidance, needed on mobile robots such as drones or self driving cars.

%\section{Comparison Algorithms} \label{sec:comparison}
\subsection{Metrics}
% & $\ddot{\bar \xi} \, [m2/s2]$ & $N_{pen} \, [\%]$ & $T_{pen} []$ & $T_{conv} [s]$ & $D_{conv} [m]$ & $\hat E_{conv} [J/kg]$ & $T_{comp} [ms]$ \\ \hline
Various metrics are defined in order to quantify the performance of the different obstacle avoidance algorithms based on DS.

\subsubsection{Square of the Relative Change of the DS}
The square of the relative change ($SRC$) gives information on how much the modulated DS $\dot{\bar \xi}$ changed compared to the original DS $\dot \xi$. It is evaluated at several points in space, before it is summed over dimensions $j$ and the samples $i$:
\begin{equation}
SRC = \sum_i \sum_j \left( \dot{\hat{\xi}}_j(\xi) - \dot{\xi}_j(\xi) \right)^2
\end{equation}
In general, the original DS should be modulated as little as possible, while avoiding the obstacle without collision. A low $SRC$ is therefore preferred.

\subsubsection{Penetration Time and Occurrence} This metric observes trajectories based on different starting points during a simulation. The penetration occurrence $N_{pene}$ indicates the percentage of trajectories which collide with the object's boundary; the penetration time $T_{pen}$ shows for how many time steps a penetrating trajectory stayed inside an object's boundary on average. (It is 0 if no penetration occurred.)

\subsubsection{Convergence Time, Distance and Relative Energy}
This metric is used in linear and stable DS with a single attractor, but could be extended to limit cycles or DS with several attractors. Convergence is reached when a trajectory is within $r_\epsilon$ of the stable attractor for several consecutive time steps. The convergence time $T_{conv}$ is the simulation time when the robot reaches convergence and the convergence distance is the sum of the discrete distance steps of its trajectory until convergence is reached:
\begin{equation}
D_{conv} = \sum_{i=1}^{i_{conv}} \left| \xi_i -\xi_{i-1} \right|
\end{equation}
where $i_{conv}$ is the simulation step at convergence. \\
The convergence relative energy is the energy it takes to reach the convergence region. It is evaluated as the discrete acceleration multiplied with the distance integrated over time. Where the acceleration is the numerical derivative of the velocity:
\begin{equation}
\hat E_{conv} = \sum_i \ddot{\xi}(\xi_i) \cdot \Delta \xi_i = \sum_i \frac{\dot{\xi}(\xi_i) - \dot{\xi}(\xi_{i-1})}{\Delta t} \cdot \left( \xi_i - \xi_{i-1} \right)
\end{equation}

\subsubsection{Computational Cost, Memory usage and Complexity}
An important aspect of an online algorithm is its computational cost. While it is hard to compare the different operations and algorithms theoretically, the computational time $T_{comp} [ms]$ can easily be measured during the execution. The computational time is expected to increase with increasing computational cost. \\
Memory usage of the algorithm is not observed, but it is expected to be low for all of them, because the DS are evaluated with functions and no big data sets are stored. \\
Moreover, the computational complexity is simplified by the number of code lines in the implementation of the algorithms.


\subsection{Results}
Further simulations are executed to compare the algorithms' performance both qualitatively and quantitatively in this section.

\subsubsection{Multiple Static Ellipses}
A good performance in a DS with four static obstacles is achieved with the DMM, where most trajectories converge to the global minimum (Fig.~\ref{fig:singleAttractor_severalObstacles_DMM_time0}). Only the elongated ellipse poses a problems for the algorithm as a local minimum is formed on its surface (southwest).\\
The IFD avoids the obstacles in a very natural way, the bifurcation seems to minimize the path around the obstacle. Moreover, several obstacles or narrow pathways pose no problem (Fig.~\ref{fig:singleAttractor_severalObstacles_IFD_time0}).

%The robots reach the attractor much slower, when obstacles are placed around it. \\ -- Simulation
\begin{figure}[tb]\centering
% \begin{subfigure}{.48\columnwidth} %
% \centering
% \includegraphics[width=\textwidth]{fig/singleAttractor_severalObstacles___time0.eps}
% \caption{-}
% \label{fig:singleAttractor_severalObstacles___time0}
% \end{subfigure} %
\begin{subfigure}{.48\columnwidth} %
\centering
\includegraphics[width=\textwidth]{fig/singleAttractor_severalObstacles_DMM_time0.eps}
\caption{DMM}
\label{fig:singleAttractor_severalObstacles_DMM_time0}
\end{subfigure}%
\begin{subfigure}{.48\columnwidth} %
\centering
\includegraphics[width=\textwidth]{fig/singleAttractor_severalObstacles_IFD_time0.eps}
\caption{IFS}
\label{fig:singleAttractor_severalObstacles_IFD_time0}
\end{subfigure}
\caption{Obstacle avoidance of four static obstacle placed in a DS with a single attractor.}
\label{fig:singleAttractor_severalObstacles_time}
\end{figure}

%%% Local Variables:
%%% mode: latex
%%% TeX-master: "../main"
%%% End:


\subsubsection{Fast Translational Moving Object}
When encountering a fast translational moving obstacle, the IFD avoids the obstacle with splitting the linear DS close to the trajectory which points through the center of mass of the obstacle (Fig.~\ref{fig:fastMovingEllipse_IFD_time13}). Furthermore,  trajectories passing in front of the moving obstacle have a larger safety margin, while the  DS in the wake of the velocity is less disturbed. \\
With the DMM, a large part of the trajectories which start behind the obstacle, as seen from the attractor, converge to a local minimum on the ellipse, instead of converging to the desired attractor (Fig.~\ref{fig:fastMovingEllipse_DMM_time13}).

\begin{figure}[tb]\centering
% \begin{subfigure}{.48\columnwidth} %
% \centering
% \includegraphics[width=\textwidth]{fig/fastMovingEllipse___time13.eps}
% \caption{-}
% \label{fig:fastMovingEllipse___time13}
% \end{subfigure} %
\begin{subfigure}{.48\columnwidth} %
\centering
\includegraphics[width=\textwidth]{fig/fastMovingEllipse_DMM_time13.eps}
\caption{DMM}
\label{fig:fastMovingEllipse_DMM_time13}
\end{subfigure}%
\begin{subfigure}{.48\columnwidth} %
\centering
\includegraphics[width=\textwidth]{fig/fastMovingEllipse_IFD_time13.eps}
\caption{IFD}
\label{fig:fastMovingEllipse_IFD_time13}
\end{subfigure}
\caption{Obstacle avoidance of a fast translational moving obstacle placed in a DS with a single attractor.}
\label{fig:fastMovingEllipse_fig}
\end{figure}



% \begin{figure}[H]\centering
% \begin{subfigure}{.24\textwidth} %
% \centering
% \includegraphics[width=\textwidth]{fig/fastMovingEllipse_none_fig0}
% \caption{}
% \label{fig:fastMovingEllipse_none_fig0}
% \end{subfigure}\,\, %
% \begin{subfigure}{.24\textwidth} %
% \centering
% \includegraphics[width=\textwidth]{fig/fastMovingEllipse_none_fig1}
% \caption{}
% \label{fig:fastMovingEllipse_none_fig1}
% \end{subfigure}
% \begin{subfigure}{.24\textwidth} %
% \centering
% \includegraphics[width=\textwidth]{fig/fastMovingEllipse_none_fig2}
% \caption{}
% \label{fig:fastMovingEllipse_none_fig2}
% \end{subfigure}
% \begin{subfigure}{.24\textwidth} %
% \centering
% \includegraphics[width=\textwidth]{fig/rotating_circle_none_fig3} % other one converged to fast
% \caption{}
% \label{fig:fastMovingEllipse_none_fig2}
% \end{subfigure}

% \begin{subfigure}{.24\textwidth} %
% \centering
% \includegraphics[width=\textwidth]{fig/fastMovingEllipse_ellipsoid_fig0}
% \caption{}
% \label{fig:fastMovingEllipse_ellipsoid_fig0}
% \end{subfigure}\,\, %
% \begin{subfigure}{.24\textwidth} %
% \centering
% \includegraphics[width=\textwidth]{fig/fastMovingEllipse_ellipsoid_fig1}
% \caption{}
% \label{fig:fastMovingEllipse_ellipsoid_fig1}
% \end{subfigure}
% \begin{subfigure}{.24\textwidth} %
% \centering
% \includegraphics[width=\textwidth]{fig/fastMovingEllipse_ellipsoid_fig2}
% \caption{}
% \label{fig:fastMovingEllipse_ellipsoid_fig2}
% \end{subfigure}
% \begin{subfigure}{.24\textwidth} %
% \centering
% \includegraphics[width=\textwidth]{fig/fastMovingEllipse_ellipsoid_fig3}
% \caption{}
% \label{fig:fastMovingEllipse_ellipsoid_fig3}
% \end{subfigure}

% \begin{subfigure}{.24\textwidth} %
% \centering
% \includegraphics[width=\textwidth]{fig/fastMovingEllipse_fluid_fig0}
% \caption{}
% \label{fig:fastMovingEllipse_fluid_fig0}
% \end{subfigure}\,\, %
% \begin{subfigure}{.24\textwidth} %
% \centering
% \includegraphics[width=\textwidth]{fig/fastMovingEllipse_fluid_fig1}
% \caption{}
% \label{fig:fastMovingEllipse_fluid_fig1}
% \end{subfigure}
% \begin{subfigure}{.24\textwidth} %
% \centering
% \includegraphics[width=\textwidth]{fig/fastMovingEllipse_fluid_fig2}
% \caption{}
% \label{fig:fastMovingEllipse_fluid_fig2}
% \end{subfigure}
% \begin{subfigure}{.24\textwidth} %
% \centering
% \includegraphics[width=\textwidth]{fig/fastMovingEllipse_fluid_fig3}
% \caption{}
% \label{fig:fastMovingEllipse_fluid_fig3}
% \end{subfigure}

% \begin{subfigure}{.24\textwidth} %
% \centering
% \includegraphics[width=\textwidth]{fig/fastMovingEllipse_rotation_fig0}
% \caption{}
% \label{fig:fastMovingEllipse_rotation_fig0}
% \end{subfigure}\,\, %
% \begin{subfigure}{.24\textwidth} %
% \centering
% \includegraphics[width=\textwidth]{fig/fastMovingEllipse_rotation_fig1}
% \caption{}
% \label{fig:fastMovingEllipse_rotation_fig1}
% \end{subfigure}
% \begin{subfigure}{.24\textwidth} %
% \centering
% \includegraphics[width=\textwidth]{fig/fastMovingEllipse_rotation_fig2}
% \caption{}
% \label{fig:fastMovingEllipse_rotation_fig2}
% \end{subfigure}
% \begin{subfigure}{.24\textwidth} %
% \centering
% \includegraphics[width=\textwidth]{fig/fastMovingEllipse_rotation_fig3}
% \caption{}
% \label{fig:fastMovingEllipse_rotation_fig3}
% \end{subfigure}
% \caption{Object avoidance one fast moving ellipse with a linear attractor DS with DMM (e - h), IFD (i - l) and LRS (m - p) compared to the dynamical system without any objects (a - d)}
% \label{fig:fastMovingEllipse_fig}
% \end{figure}

%%% Local Variables:
%%% mode: latex
%%% TeX-master: "../main"
%%% End:


\subsubsection{Object moving close to origin.}
Fig.~\ref{fig:leftMoving_circle_fig} shows the behavior of the two algorithms, when a circular obstacle is moving close to the attractor position. For the IFD, the global minimum is shifted slightly away in direction opposite to the direction of the obstacle (southwest), this seems to be a good strategy to get a larger safety margin when the obstacle passes. On the other hand, it is not necessary to move, because the obstacle is not going to hit the attractor.\\
The global minimum with the DMM is also shifted to ensure no collision with the approaching obstacle. It is shifted in direction southeast, which is a less optimal option. Additionally, it is shifted less than with the IFD.

\begin{figure}[tb]\centering
% \begin{subfigure}{.48\columnwidth} %
% \centering
% \includegraphics[width=\textwidth]{fig/leftMoving_circle___time20.eps}
% \caption{-}
% \label{fig:leftMoving_circle___time2}
% \end{subfigure} %

\begin{subfigure}{.48\columnwidth} %
\centering
\includegraphics[width=\textwidth]{fig/leftMoving_circle_DMM_time20.eps}
\caption{DMM}
\label{fig:leftMoving_circle_DMM_time2}
\end{subfigure}%
\begin{subfigure}{.48\columnwidth} %
\centering
\includegraphics[width=\textwidth]{fig/leftMoving_circle_IFD_time20.eps}
\caption{IFD}
\label{fig:leftMoving_circle_IFD_time2}
\end{subfigure}
\caption{Obstacle avoidance when moving close to the origin in a DS with a single attractor.}
\label{fig:leftMoving_circle_fig}
\end{figure}

% \begin{figure}[H]\centering
% \begin{subfigure}{.32\textwidth} %
% \centering
% \includegraphics[width=\textwidth]{fig/leftMoving_circle_ellipsoid_fig2}
% \caption{}
% \label{fig:leftMoving_circle_ellipsoid_fig2}
% \end{subfigure}\,\, %
% \begin{subfigure}{.32\textwidth} %
% \centering
% \includegraphics[width=\textwidth]{fig/leftMoving_circle_ellipsoid_fig3}
% \caption{}
% \label{fig:leftMoving_circle_ellipsoid_fig3}
% \end{subfigure}
% \begin{subfigure}{.32\textwidth} %
% \centering
% \includegraphics[width=\textwidth]{fig/leftMoving_circle_ellipsoid_fig4}
% \caption{}
% \label{fig:leftMoving_circle_ellipsoid_fig4}
% \end{subfigure}


% \begin{subfigure}{.32\textwidth} %
% \centering
% \includegraphics[width=\textwidth]{fig/leftMoving_circle_fluid_fig2}
% \caption{}
% \label{fig:leftMoving_circle_fluid_fig2}
% \end{subfigure}\,\, %
% \begin{subfigure}{.32\textwidth} %
% \centering
% \includegraphics[width=\textwidth]{fig/leftMoving_circle_fluid_fig3}
% \caption{}
% \label{fig:leftMoving_circle_fluid_fig3}
% \end{subfigure}
% \begin{subfigure}{.32\textwidth} %
% \centering
% \includegraphics[width=\textwidth]{fig/leftMoving_circle_fluid_fig4}
% \caption{}
% \label{fig:leftMoving_circle_fluid_fig4}
% \end{subfigure}

% \begin{subfigure}{.32\textwidth} %
% \centering
% \includegraphics[width=\textwidth]{fig/leftMoving_circle_rotation_fig2}
% \caption{}
% \label{fig:leftMoving_circle_rotation_fig2}
% \end{subfigure}\,\, %
% \begin{subfigure}{.32\textwidth} %
% \centering
% \includegraphics[width=\textwidth]{fig/leftMoving_circle_rotation_fig3}
% \caption{}
% \label{fig:leftMoving_circle_rotation_fig3}
% \end{subfigure}
% \begin{subfigure}{.32\textwidth} %
% \centering
% \includegraphics[width=\textwidth]{fig/leftMoving_circle_rotation_fig4}
% \caption{}
% \label{fig:leftMoving_circle_rotation_fig4}
% \end{subfigure}
% \caption{Object avoidance of one circle which only starts moving after the solution trajectories have almost converged to the attractor with  DMM(a - c), IFD (d - f) and LRS (g - i)}
% \label{fig:leftMoving_circle_fig}
% \end{figure}

%%% Local Variables:
%%% mode: latex
%%% TeX-master:
%%% "../main"
%%% End:


\subsubsection{Single Rotating Object}
The simulation of a rotating ellipse close to the position of the attractor yields (Fig.~\ref{fig:rotating_circle_fig}), that with DMM many trajectories turn in the mathematically positive way (same sense as the obstacle). Furthermore many trajectories move to a local minimum on the obstacles boundary (east). Apart from that, the convergence of the other trajectories works well. \\
The IFD has also many paths turning in the mathematically positive way, but more paths turn the in the negative sens than with IFD. Likewise, a large avoidance region can be observed south of the obstacle, where a fast rotational velocity is expected. Southwest of the obstacle, where it is rotating away from the DS, a \textif{sucking} effect is observed. The trajectories are pulled back to the obstacle, which seems to be analogous to an incompressible fluid. \\
\begin{figure}[tb]\centering
% \begin{subfigure}{.48\columnwidth} %
% \centering
% \includegraphics[width=\textwidth]{fig/rotating_ellipse___time28.eps}
% \caption{-}
% \label{fig:rotating_ellipse___time0}
% \end{subfigure} %
\begin{subfigure}{.48\columnwidth} %
\centering
\includegraphics[width=\textwidth]{fig/rotating_ellipse_DMM_time28.eps}
\caption{DMM}
\label{fig:rotating_ellipse_DMM_time0}
\end{subfigure}%
\begin{subfigure}{.48\columnwidth} %
\centering
\includegraphics[width=\textwidth]{fig/rotating_ellipse_IFD_time28.eps}
\caption{IFD}
\label{fig:rotating_ellipse_IFD_time0}
\end{subfigure}
\caption{Obstacle avoidance of a rotating ellipse in a DS with a single attractor (to direction and magnitude of the rotation is visualized with the red arrow).}
\label{fig:rotating_circle_fig}
\end{figure}

% \begin{figure}[H]\centering
% \begin{subfigure}{.24\textwidth} %
% \centering
% \includegraphics[width=\textwidth]{fig/rotating_circle_none_fig0}
% \caption{}
% \label{fig:rotating_circle_none_fig0}
% \end{subfigure}\,\, %
% \begin{subfigure}{.24\textwidth} %
% \centering
% \includegraphics[width=\textwidth]{fig/rotating_circle_none_fig1}
% \caption{}
% \label{fig:rotating_circle_none_fig1}
% \end{subfigure}
% \begin{subfigure}{.24\textwidth} %
% \centering
% \includegraphics[width=\textwidth]{fig/rotating_circle_none_fig2}
% \caption{}
% \label{fig:rotating_circle_none_fig2}
% \end{subfigure}
% \begin{subfigure}{.24\textwidth} %
% \centering
% \includegraphics[width=\textwidth]{fig/rotating_circle_none_fig2}
% \caption{}
% \label{fig:rotating_circle_none_fig3}
% \end{subfigure}

% \begin{subfigure}{.24\textwidth} %
% \centering
% \includegraphics[width=\textwidth]{fig/rotating_circle_ellipsoid_fig0}
% \caption{}
% \label{fig:rotating_circle_ellipsoid_fig0}
% \end{subfigure}\,\, %
% \begin{subfigure}{.24\textwidth} %
% \centering
% \includegraphics[width=\textwidth]{fig/rotating_circle_ellipsoid_fig1}
% \caption{}
% \label{fig:rotating_circle_ellipsoid_fig1}
% \end{subfigure}
% \begin{subfigure}{.24\textwidth} %
% \centering
% \includegraphics[width=\textwidth]{fig/rotating_circle_ellipsoid_fig2}
% \caption{}
% \label{fig:rotating_circle_ellipsoid_fig2}
% \end{subfigure}
% \begin{subfigure}{.24\textwidth} %
% \centering
% \includegraphics[width=\textwidth]{fig/rotating_circle_ellipsoid_fig3}
% \caption{}
% \label{fig:rotating_circle_ellipsoid_fig3}
% \end{subfigure}

% \begin{subfigure}{.24\textwidth} %
% \centering
% \includegraphics[width=\textwidth]{fig/rotating_circle_fluid_fig0}
% \caption{}
% \label{fig:rotating_circle_fluid_fig0}
% \end{subfigure}\,\, %
% \begin{subfigure}{.24\textwidth} %
% \centering
% \includegraphics[width=\textwidth]{fig/rotating_circle_fluid_fig1}
% \caption{}
% \label{fig:rotating_circle_fluid_fig1}
% \end{subfigure}
% \begin{subfigure}{.24\textwidth} %
% \centering
% \includegraphics[width=\textwidth]{fig/rotating_circle_fluid_fig2}
% \caption{}
% \label{fig:rotating_circle_fluid_fig2}
% \end{subfigure}
% \begin{subfigure}{.24\textwidth} %
% \centering
% \includegraphics[width=\textwidth]{fig/rotating_circle_fluid_fig3}
% \caption{}
% \label{fig:rotating_circle_fluid_fig3}
% \end{subfigure}

% \begin{subfigure}{.24\textwidth} %
% \centering
% \includegraphics[width=\textwidth]{fig/rotating_circle_rotation_fig0}
% \caption{}
% \label{fig:rotating_circle_rotation_fig0}
% \end{subfigure}\,\, %
% \begin{subfigure}{.24\textwidth} %
% \centering
% \includegraphics[width=\textwidth]{fig/rotating_circle_rotation_fig1}
% \caption{}
% \label{fig:rotating_circle_rotation_fig1}
% \end{subfigure}
% \begin{subfigure}{.24\textwidth} %
% \centering
% \includegraphics[width=\textwidth]{fig/rotating_circle_rotation_fig2}
% \caption{}
% \label{fig:rotating_circle_rotation_fig2}
% \end{subfigure}
% \begin{subfigure}{.24\textwidth} %
% \centering
% \includegraphics[width=\textwidth]{fig/rotating_circle_rotation_fig2}
% \caption{}
% \label{fig:rotating_circle_rotation_fig2}
% \end{subfigure}
% \caption{Object avoidance of a rotating ellipse obstacle by a linear attractor DS with DMM (e - h), IFD (i - l) and LRS (m - p) compared to the dynamical system without any objects (a - d)}
% \label{fig:rotating_circle_fig}
% \end{figure}

%%% Local Variables:
%%% mode: latex
%%% TeX-master: "../main"
%%% End:


\subsubsection{Quantitative comparison}
For the quantitative comparison, a set of initial conditions is chosen and the simulation is run with either IFD or DMM until convergence or the time limit is reached. \\ %
According to the SRC measure the IFD modifies the DS less (Tab.~\ref{tab:twoMovingRotatingObs} -\ref{tab:fastMovingEllipse}), as it has a lower value of at least 15\%  compared to the DMM. \\
The IFD tends to have more collisions, with up to 13\% of the trajectories colliding, while the DMM never collides with an obstacle (Tab.~\ref{tab:fastMovingEllipse}). \\
The convergence time $T_{conv}$ is in the same range for IFD and DMM (Tab.~\ref{tab:fourStaticObjects}).The path length $D_{conv}$ and the relative convergence energy $\hat E_{conv}$ on the other hand are lower for the IFD during the three simulations. The DMM slightly lags having around 10\%-20\% higher convergence distance and up to 100\% higher relative convergence energy. In general, the energy seems to correlate with the value of the SRC. \\
The computational time for the IFD and the DMM are in the same range and vary only by about 10\%: the IFD has a higher computational time with several obstacles.  \\
The complexity based on the length of the code is comparable for both algorithms with around 160 ($\pm$20) lines.

\begin{table}[h]
\centering
\caption{Simulation metrics of a dynamical system with two moving obstacles with time step $dt = 0.005 s$, maximum iteration $i_{max} = 1200$ (similar to Fig.~\ref{fig:twoMovingRotatingObs_DMM} \& \ref{fig:twoMovingRotatingObs_IFD}).}
\label{tab:twoMovingRotatingObs}
\begin{tabular}{|l|r|r|r|} \hline
 &\multicolumn{1}{c|}{-} &\multicolumn{1}{c|}{DMM} &\multicolumn{1}{c|}{IFD} \\ \hline
SRC $[m2/s2]$ &0.0000 &197.6862 &166.6479 \\ \hline
$N_{pen} \, [\%]$ &0.0 &0.0 &0.0 \\ \hline
$T_{pen} []$ &0 &0 &0 \\ \hline
$T_{conv} [s]$ &5.3050 &5.9900 &5.9900 \\ \hline
$D_{conv} [m]$ &22.1793 &25.7054 &24.0317 \\ \hline
$\hat E_{conv} [J/kg]$ &25.3305 &87.8503 &61.5909\\ \hline
$T_{comp} [ms]$ &0.0000 &0.0113 &0.0116 \\ \hline
\end{tabular}
\end{table}


\begin{table}[h]
\centering
\caption{Simulation metrics of a dynamical system with four static objects  with time step $dt = 0.01 s$, maximum iteration $i_{max} = 1200$ and convergence tolerance $r_\epsilon = 0.1 m$ (similar to Fig.~\ref{fig:singleAttractor_severalObstacles_time}).}
\label{tab:fourStaticObjects}
\begin{tabular}{|l|r|r|r|} \hline
 &\multicolumn{1}{c|}{-} &\multicolumn{1}{c|}{DMM} &\multicolumn{1}{c|}{IFD} \\ \hline
SRC $[m2/s2]$ &0.0000 &129.2555 &99.4898 \\ \hline
$N_{pen} \; [\%]$ &0.0 &0.0 &3.3 \\ \hline
$T_{pen} [-] $ &0 &0 &2 \\ \hline
$T_{conv} [s]$ &4.6250 &9.6500 &9.7750 \\ \hline
$D_{conv} [m]$ &9.6599 &12.1595 &11.6221 \\ \hline
$\hat E_{conv} [J/kg]$ &37.0222 &127.2202 &55.9537 \\ \hline
$T_{comp} [ms]$ &0.0000 &0.0096 &0.0102 \\ \hline
\end{tabular}
\end{table}



\begin{table}[h]
\centering
\caption{Simulation metrics of a dynamical system with a fast moving ellipse (similar to Fig.~\ref{fig:fastMovingEllipse_fig}) with time step $dt = 0.005 s$, maximum iteration $i_{max} = 1200$ and convergence tolerance $r_\epsilon = 0.1 m$ .}
\label{tab:fastMovingEllipse}
\begin{tabular}{|l|r|r|r|} \hline
 &\multicolumn{1}{c|}{-} &\multicolumn{1}{c|}{DMM} &\multicolumn{1}{c|}{IFD} \\ \hline
SRC $[m2/s2]$ &0.00 &665.40 &322.58 \\ \hline
$N_{pen} \, [\%]$ &0.0 &0.0 &13.3\\ \hline
$T_{pen} [-]$ &0 &0 &54 \\ \hline
$T_{conv} [s]$ &5.3050 &5.9900 &5.8150 \\ \hline
$D_{conv} [m]$ &22.2363 &31.9179 &26.5775 \\ \hline
$\hat E_{conv} [J/kg]$ &19.1033 &175.4631 &109.4048 \\ \hline
$T_{comp} [ms]$ &0.0000 &0.0079 &0.0072 \\ \hline
\end{tabular}
\end{table}

% \begin{table}[h]
% \centering
% \caption{Complexity of the algorithm is analyzed as the number of lines of acting code (without comments and backspaces) used to express the obstacle avoidance. }
% \label{tab:computationalComplexity}
% \begin{tabular}{|l|r|} \hline
% & Length of code [\# of lines]\\ \hline
% DMM & 143 \\ \hline
% IFD & 183 \\ \hline
% LRS & 169 \\ \hline
% \end{tabular}
% \end{table}




%%% Local Variables:
%%% mode: latex
%%% TeX-master: "../main"
%%% End:


%%% Local Variables:
%%% mode: latex
%%% TeX-master: "../main"
%%% End:


\section{Discussion} \label{sec:discussion}
%\subsection{DMM}

% While the DMM avoids the obstacle without collision, it often chooses a non-intuitive path to avoid them. Additionally, these paths demand more changes to the initial DS than the one chosen by the IFD. In some cases, the DMM modifies the DS to converge to a local minimum on the obstacle boundary, rather than avoiding it. In these cases the DMM switches to a \textit{contouring}-mode and safely evades these local minima, but this is not an optimal way to avoid the object \cite{khansari2012dynamical}. Additionally, this is not a closed form solution.
% The algorithm has not been extended to concave object. However some propositions exist for obstacles in discrete space represented by point clouds \cite{saveriano2013point}. \\
% %\subsection{IFD}
% The IFD has a lower value for the convergence distance, time and relative energy. Additionally, it modifies the original DS less (lower $ SRC $-value). On the other hand, it shows several collisions with the obstacle. Collisions occurred for the IFD in simulations with either several and close,  or fast moving and rotating obstacles. This is however expected to occur due to numerical issues or errors in the implementation, because stability could be proven (Sec.~\ref{sec:IFD}). An indication of numerical issues is the decreasing number of collisions with shorter time steps.

%While the two algorithms, IFD and DMM, have different procedure, the math behind them is  similar. They both use a modulation matrix with similar eigenvalues which is applied to reduce the direction of the normal but keep tangential speed. The
On the one hand, harmonic functions, which the IRF and DMM are based on, have the property, that equilibrium points only occur on the boundaries of the space \cite{kim1992real} On the other hand, a smooth vector field in a sphere world must have at least as many saddle points as obstacles \cite{koditschek1990robot}. As a result of this, these extrema must occur either on the obstacle boundary or the space limits. Such a saddle point can be observed on a circle in a stream function, where the stream line directly ahead of the obstacle will converge to the saddle point on the circle's surface. \\
However, the DMM shows convergence of not just one but several trajectories to a locally stable point (local minimum). This problem of local minima does not appear for the IFD where only one trajectory converges to the saddle point, as it can be observed for a circular obstacle in two dimensional flow. \\
The reason for this difference becomes clear when observing the base vectors used for the modulation (Fig.~\ref{fig:comparison_behaviour}). While the tangent (green) is the same with the IFD and the DMM, the normal (red) differs. However the normal and tangent create an independent set of vectors in both cases. \\
The consequence of this can be observed by analyzing a specific point, e.g. (-1,-1.5) in Fig.~\ref{fig:avoidingEllipse_IFD_ERF}. We can see that the gradient for the corresponding normal (green) is steeper than the gradient of the DS (gray) for the IFD. As a result, for a linear combination based on the normal and tangent a positive value in direction of the tangent (green) is needed. Conversely, the gradient of the normal (red) is less steep for the DMM than the IFD, therefore a negative value in direction of the tangent (green) is needed for the linear combination of the original DS based on the tangent and normal. Close to the obstacle, the magnitude of the DS in direction of the normal is (partially) removed, as a result the modulated DS has a positive value in direction of the tangent for the IRF, but a negative value in direction of the tangent for the DMM. For the IRF the DS avoids the obstacle safely, while for the DMM it converges on the local minimum in front of it. \\

For a linear DS with single attractor the DMM creates a local minimum on the ellipse boundary if it is placed in a bad way, as could be observed in several simulations (Sec.~\ref{sec:DMM}). On the other hand, the IFD creates a saddle point on the obstacles surface where only one trajectory converges to. The rest of the DS is safely modulated to converge to the attractor. This is clearly a better performance of the IFD compared to the DMM (One could still think of some nonlinear DS, where even the IFD creates a local minimum on the surface. Such cases are not treated in this report.)


\subsection{Limitations}
Most of the simulation in this report are performed using a linear DS with a single attractor, because its dynamics are simple and the modulation can be easily observed. For general performance analysis of the algorithms the algorithms should be applied to more complex, nonlinear DS. \\

\subsection{Future Work}
Future work should focus on the extension of the algorithms to concave objects. Even if the robots try to avoid those regions by creating a convex hull around them, a situation might occur where a concave region is formed by closely moving obstacles and the robots ends up in it. While there have been solutions proposed by \cite{saveriano2013point,}, they are for point clouds and no analytic solution exists.\\
Furthermore, several moving obstacles should be observed and an algorithm to solve this should be implemented. \\
Another important aspect to test stability is to add  noise and disturbances to the simulations. This allows to judge the stability of the algorithms and if it is implementable on real robots. \\
The implementation of the algorithm on the ROS simulator and testing on real robots should be done to finally conclude about the performance of the algorithms. This has been done for the DMM \cite{khansari2012dynamical}. \\
While the metrics give a good inside on the behavior of the algorithm, it should be evaluated which one is the most important for the robot's performance. Moreover, it should be evaluated how an algorithm should behave in extreme situations, e.g. when an obstacle is passing by close or being close close to a rotating obstacle.
%\subsubsection{IFD}
Different options for the choice of the eigenvectors $\labmda_i$ could be further observed. \\
The reason for the frequent collisions should be further observed and a method should be found to ensure impenetrability in discrete space. \\
Any future work should consult \cite{feder1997real,khansari2012realtime,khansari2012dynamical,saveriano2013point,saveriano2014distance,}, because those obstacle avoidance approaches use similar fluid dynamic based algorithms. Furthermore, they proposed and tested several methods for moving and concave methods.

\section{Conclusion}
%While there is no clear winner between the algorithms. The IFD seems to be the fastest algorithm, which does change the DS the least among the three. The DMM shows slightly more stable behavior, but has also fast converging solution. On the other hand, LRS seems to perform worse, with abrupt changes for implementation, lacking of convergence and the missing closed form makes it unfit for dynamical systems.  \\
%Even though has the most collisions, we prefer it over the other algorithms and promote its implementation as it shows the most intuitive trajectory choice. The DMM on the other hand often converges to a local minimum in front of an obstacle. The LRS is with its current set of hyper-parameters not usable for obstacle avoidance with several obstacles an nonlinear DS.
The IFD has clear advantage over the DMM for linear DS sytem with a single attractor which are often used in pick-and-place actions in robotics. While the DMM tends to create local minima on the obstacle boundary for ellipses, the IFD only has a saddle point for one trajectory and all but one of its trajectories converge. The strategy of not using orthogonal base vectors for the application of the modulation matrix shows a large improvement to the DMM. On the other hand, avoiding several, moving and concave obstacles need to be further observed,

%%% Local Variables:
%%% mode: latex
%%% TeX-master: "../main"
%%% End:


% \section{CONCLUSIONS}
% A conclusion section is not required. Although a conclusion may review the main points of the paper, do not replicate the abstract as the conclusion. A conclusion might elaborate on the importance of the work or suggest applications and extensions.


\addtolength{\textheight}{-0cm}   % This command serves to balance the column lengths
                                  % on the last page of the document manually. it shortens
                                  % the textheight of the last page by a suitable amount.
                                  % this command does not take effect until the next page
                                  % so it should come on the page before the last. make
                                  % sure that you do not shorten the textheight too much.

                                  %%%%%%%%%%%%%%%%%%%%%%%%%%%%%%%%%%%%%%%%%%%%%%%%%%%%%%%%%%%%%%%%%%%%%%%%%%%%%%%%
                                  % \section*{appendix}
\appendix
\subsection{Proof of Theorem 1} \label{sec:proof1}
 %\textbf{Impenetrability}
%Any coordinate transformation $\mathcal{E} \to \mathcal{C}:  \mathbb{R}^d \to \mathbb{R}^d$ which transforms a convex region in $\mathcal{X}_1^d \subset \mathcal{E}$ to a convex region in $\mathcal{X}_2^d \subset \mathcal{C}$, ensures that any vector starting and ending in the convex region $\dot \xi_{\mathcal{E}} \in \mathcal{X}_1$ will be contained in the convex region $\dot \xi_{\mathcal{C}} \in \mathcal{X}_2$, vise versa any vector outside the starting and ending outside will start and end outside this region.

%The linear transformation with the stretching matrix $A$ and rotation matrix $R$, as well as the translation with $\xi_0$ keep the convex regions intact. As the velocity vector starts and stays outside the concave circle and will therefore stay that way for the transformation back to the ellipse space, while impenetrability is  given. \hfill $\blacksquare$ \vspace{1em}

% The surface of the ellipse can be described as:
% \begin{equation}
%  \frac {{}_{e}\xi_1 ^ 2}{a^2} + \frac{{}_e\xi_2^2}{b^2} = 1
% \end{equation}
The subscript ${}_e (\cdot)$ declares that a variable is given in the ellipse reference frame (ERF).
With the Jacobian of the level function of an ellipse described in (\ref{eq:levelEllipse}), a normal ${}_e \vec n_e (\xi)$ on the ellipse can be found:
\begin{equation}
  {}_e \vec{n}_{e} (\xi) =
  \frac{1}{\sqrt{a_2^4 \cdot {}_e\xi_1^2 + a_1^4 \cdot{}_e \xi_2^2}}
  \begin{bmatrix}
    a_2^2 \cdot {}_e\xi_1 \\
    a_1^2 \cdot {}_e\xi_2
  \end{bmatrix}
\end{equation}

The unit tangent on the ellipse ${}_e\vec e_{e}$ is perpendicular to the normal, and can be written as:
\begin{equation}
  {}_e \vec{e}_{e} (\xi)=
  \frac{1}{\sqrt{a_2^4 \cdot {}_e\xi_1^2 + a_1^4 \cdot{}_e \xi_2^2}}
  \begin{bmatrix}
    a_1^2 \cdot {}_e\xi_2 \\
    - a_2^2 \cdot {}_e\xi_1
  \end{bmatrix}
\end{equation}

Before the modulation is applied, the whole coordinate system is transformed, such that the ellipse lies on a unit circle, which is referred as circle reference frame (CRF) with the subscript ${}_c(\cdot)$:
\begin{equation}
  %{}_c \xi = A^{-1} {}_e \xi \quad \rightarrow \quad
  {}_e \xi_1 (\xi) = a_1 \cdot {}_c\xi_1 \, , \;\; {}_e \xi_2 = a_2 \cdot {}_c\xi_2 \label{eq:ellipseCircleTrafo}
\end{equation}
%For a unit circle, the two axis have the same value, and are given by $r = a = b = 1$. \\
% \begin{equation}
%  \xi_1 ^ 2 + \xi_2^2 = 1
% \end{equation}
The normal ${}_c \vec n _{c} (\xi)$ of the unit circle in CRF can be evaluated similarly with the level function (\ref{eq:levelCircle})
\begin{equation}
  {}_c \vec{n}_{c} (\xi) =
  \frac{1}{\sqrt{ {}_c\xi_1^2 + {}_c \xi_2^2}}
  \begin{bmatrix}
     {}_c\xi_1 \\
     {}_c\xi_2
  \end{bmatrix}
\end{equation}

and the tangent
\begin{equation}
  {}_c \vec{e}_{c} (\xi) =
  \frac{1}{\sqrt{ {}_c\xi_1^2 + {}_c \xi_2^2}}
  \begin{bmatrix}
     {}_c\xi_2 \\
     - {}_c\xi_1
  \end{bmatrix}
\end{equation}

On the obstacle surface, the velocity needs to be solely in tangential direction. While this was shown for the circle, the proof has to be extended to the ellipse. For this, the tangent on the unit circle needs to be equal to the tangent of the ellipse, too. To evaluate this, the tangent on the unit circle in CRF is transformed to the ERF:
\begin{equation}
  {}_e \vec{e}_{c} (\xi) n= \begin{bmatrix} a_1 & 0 \\ 0 & a_2\end{bmatrix} \cdot {}_c \vec{e}_{c}
   \qquad
   {}_e \vec{e}_{c} = \frac{1}{\sqrt{a^2 \cdot {}_c\xi_2^2 + a_2^2 {}_c\xi_1^2}}
  \begin{bmatrix}
       a_1 \cdot {}_c\xi_2 \\
     - a_2 \cdot {}_c\xi_1
   \end{bmatrix}
\end{equation}
Furthermore, the coordinate transformation can be applied (Eq.~\ref{eq:ellipseCircleTrafo}) and leads to:
\begin{equation}
  {}_e \vec{e}_{c} (\xi) =
  \frac{1}{\sqrt{a_2^4 \cdot {}_e\xi_1^2 + a^4 \cdot{}_e \xi_2^2}}
  \begin{bmatrix}
    a_1^2 \cdot {}_e\xi_2 \\
    - a_2^2 \cdot {}_e\xi_1
  \end{bmatrix}
   =
    {}_e \vec{e}_{e} (\xi)
\end{equation}
which is equal to the tangent on the ellipse in ellipse space. Projecting the velocity solely on the tangent in ellipse space is ensured and therefore impenetrability is guaranteed. \hfill $\blacksquare$ \\

It is important to notice, that the normal on the circle transformed to the ERF ${}_e \vec n_{c} $ is not normal to the ellipse:
\begin{equation}
  {}_e\vec n_{c} (\xi)
  =
  \frac{1}{a_1 a_2\sqrt{ {}_e\xi_1^2 + {}_e \xi_2^2}}
  \begin{bmatrix}
     a_1 a_2 \cdot{}_e\xi_1 \\
     a_1 a_2 \cdot {}_e\xi_2
   \end{bmatrix}
  \neq
  {}_e\vec n_{e} (\xi)
\end{equation}
As a result of this, the normal and tangent of the CRF are not orthogonal in ERF, but linearly independent. Zero velocity in direction of the normal on the ellipse can still be ensured, because the normal on the ellipse ${}_e \vec n_{e} (\xi)$ is a linear combination of the normal on the circle in ERF (${}_e \vec n_{c} (\xi)$, ${}_e \vec e_{e}(\xi)$). As there is no velocity in direction of ${}_e \vec n_{c} (\xi) $ it is a trivial normal combination resulting in ${}_e \vec n_{e} (\xi)= 0$.




\subsection{Proof of Theorem 2} \label{sec:proof2}
The Neuman Boundary condition is met for a linearly independent $\hat{D} = [\hat n (\xi) \; e_1(\xi) ]$ base matrix with $e_1(\xi) $ the tangential hyper-plane and $\hat n$ a vector linearly independent (not necessarily the normal to the plane), if any velocity in direction of $\hat n(\xi)$ is removed for every point on the surface of a convex obstacle.

For the two dimensional case, a pseudo-normal $\hat{ \vec n}$ is defined which is inclined at an angle of $\phi \in ]-\frac{\pi}{2}, \frac{\pi}{2}[$. It can be expressed as a linear combination of the normal $\vec n$ and the tangent $\vec e_1$:
\begin{equation}
\hat{ \vec n} (\xi) = \cos \phi \cdot \vec n (\xi) + \sin \phi \cdot \vec e_1 (\xi)
\end{equation}

For any point on the surface, it could be shown that $\xi \in \mathcal(X)^b$ the modulation ensures that the velocity in direction of the normal vanishes. For our case, this normal is replaces with the pseudo-normal. The Neuman boundary can be ensured:
\begin{align}
  \dot{\vec \xi} \cdot \vec n (\xi) &=  \dot{\vec \xi} \cdot \left(\frac{1}{\cos \phi} \hat{\vec n}(\xi) - \frac{\sin \phi}{\cos \phi} \vec{e}_1 (\xi) \right) \\
  & = \left| \dot{\vec \xi}  \right| \cdot \left(\frac{\cos(\frac{\pi}{2}-\phi)}{\cos \phi} - \frac{\sin \phi}{\cos \phi}\right) = 0
\end{align}
Impenetrability is therefore given. \hfill $\blacksquare$



\subsection{Proof of Theorem 3} \label{sec:proof3}
The proof is evaluated for a two dimensional ellipse. Furthermore, without any loss of generalization, the obstacle is placed at the origin which results in $\xi^o = [0 \; : \; 0]^T$ and the linear attractor at a position $\xi^a = \begin{bmatrix} d_1 & d_2 \end{bmatrix}$ with $d_1 \in \mathbb{R}^\plus$ and $d_2 \in \mathbb{R}$. The DS for such a case is given as $f(\xi) = \begin{bmatrix} -(\xi_1-d_1) & -(\xi_2-d_2)\end{bmatrix}^T$.
The components of the modulation matrix $\hat M$ of Sec.~\ref{sec:IFD}
\begin{equation}
  D(\xi) =
  \begin{bmatrix}
    \lambda_n(\tilde \xi) & 0 \\
    0 & \lambda_1(\tilde \xi)
  \end{bmatrix}
  \qquad
  \hat{E} (\xi) =
  \begin{bmatrix}
    \hat{\vec n } & \vec e_1
  \end{bmatrix}
\end{equation}
where $\hat{\vec n} = \begin{bmatrix}$ \xi_1 & \xi_2 \end{bmatrix} and $\vec e_1 = \begin{bmatrix} a_1^2 \xi_1 & a_2^2 \xi_2 \end{bmatrix}$.\\
\begin{align}
  \hat M ( \xi) &= \hat{E}(\xi) D(\xi) \hat{E}(\xi) %\\
  %               & = \frac{1}{a^2_2 \xi_1^2 + a_1^2 \xi_2^2} \cdot \\
  % & \cdot
  % \begin{bmatrix}
  %   a_2^2 \xi_1^2 \lambda_n(\tilde \xi) + a_1^2 \xi_2^2 \lambda_{e1}(\tilde \xi)   & \xi_1\xi_2 (a_1^2 \lambda_n(\tilde \xi) + a_2^2 \lambda_{e1}(\tilde \xi) ) \\
  %   \xi_1\xi_2 (a_2^2 \lambda_n(\tilde \xi) + a_1^2 \lambda_{e1}(\tilde \xi) ) & a_1^2 \xi_1^2 \lambda_n(\tilde \xi) + a_2^2 \xi_2^2 \lambda_{e1}(\tilde \xi)
  % \end{bmatrix}
\end{align}
It can be observed that there is one trajectory that converges to a saddle point on the obstacles surfaces,  which is given by $\xi_2 = 0$. The property of the original dynamics of this trajectory defined by $\mathcal{X}^S = \{\xi \in \mathbb{R}^2 \}$. It has the property, that there is only a velocity part in direction of $\hat \vec{n}$ but none in tangential direction of the surface. \\
To ensure that no local minima exists, it need to be shown thath all other trajectories converge to the global attractor. This is done by showing, that any trajectory above (greater y direction) this saddle point trajectory $\mathcal{X}^S$ turns in the mathematical positive sens and and trajectory below turns in the opposite direction. They therefore avoid converging to the saddle point.\\

The modulated DS is given by:
\begin{align}
  \dot{\xi} &= \hat{M}(\xi) \cdot f(\xi)
  % \begin{pmatrix}
  %    \lambda_n(\tilde \xi) (-a_2^2 \xi_1^3 + a_2^2 d \xi_1^2 - a_1^2 \xi_1 \xi_2^2) +   \lambda_{e1}(\tilde \xi) a_1^2 d \xi_2^2 \\
  %    \lambda_n(\tilde \xi) (-a_2^2 \xi_1^2 \xi_2 + a_2^2 d \xi_1 \xi_2 - a_1^2 \xi_2^3) +   \lambda_{e1}(\tilde \xi) a_2^2 d \xi_1 \xi_2
  %\end{pmatrix}
\end{align}

The cross product in two defined as $\vec a \times \vec b = \Vert \vec a \Vert \Vert \vec b \Vert \sin \theta \cdot$, where $\theta$ is the relative rotation of $\vec b$ towards $\vec a$ in positive direction around normal of the two vectors. Because any point above the centerline $\mathacl{X}^c$ as give in (\ref{eq:centerLine}) and behind the obstacle ($\xi_1 \leq 0$), needs to rotate in the mathematical positive sense to move away from the center line and not create a local minimum on the obstacle. This can be expressed as:
\begin{align}
  &f(\xi) \times \dot \xi > 0 &\xi \in \{\xi \in \mathbb{R}^2: \xi_1 \leq 0,\; \xi_2 > \frac{d_2 \xi_1}{d_1}  \} \label{eq:condition_noMimima} \\
  &f(\xi) \times \dot \xi < 0 &\xi \in  \{\xi \in \mathbb{R}^2: \xi_1 \leq 0,\; \xi_2 < \frac{d_2 \xi_1}{d_1}  \} \label{eq:condition_noMimima2}
  %&f(\xi) \times \dot \xi < 0 &\mathcal{X}^{3} = \{\xi \in \mathbb{R}^2: \xi_1 \leq 0,\; y < 0 \}
\end{align}

The cross product is evaluated as:
\begin{equation} \label{eq:crossProduct}
  \begin{split}
  f(\xi) \times \dot \xi =&  \left( \lambda_n(\tilde \xi) - \lambda_{1}(\tilde \xi) \right) \frac{d_1\xi_2 - d_2 \xi_1}{a_2^2 \xi_1^2 + a_1^2 \xi_2^2} \cdot \\
  & \left(a_1^2\xi_2(\xi_2 - d_2) + a_2^2 \xi_1 (\xi_1 - d_1)  \right)
  % \left( \lambda_n(\tilde \xi) - \lambda_{e1}(\tilde \xi) \right) \left( d a_2^2 \xi_1^2 \xi_2 - a_2^2 d^2 \xi_1 \xi_2 + a_1^2 d \xi_2^2 \right)
  \end{split}
\end{equation}

First we try to show, that vector $f(\xi)$ above the centerline is rotated clockwise as described in (\ref{eq:condition_noMimima})
The second factor in (\ref{eq:crossProduct}) can be evaluated with the condition from of being above the centerline ($\xi_2 > \frac{d_2 \xi_1}{d_1}$) as:
\begin{equation}
 \frac{d_1\xi_2 - d_2 \xi_1}{a_2^2 \xi_1^2 + a_1^2 \xi_2^2} > 0
\end{equation}
Further more, with the condition of being above the centerline ($d_2 < \frac{d_1}{\xi_1} \xi_2$) and only considering points on the left hand side ($\xi_1 \leq 0$), the first part of the third factor in (\ref{eq:crossProduct}) can be observed in the three different regions of: \\
In the top half of the coordinate system with $\xi_2 > 0$, it follows:
\begin{equation}
  a_1^2\xi_2(\xi_2 - d_2) > a_1^2\xi_2(\xi_2 - \xi_2 \frac{d_1}{\xi_1}) = a_1^2 \xi_2^2(1+\frac{d_1}{\| \xi_1\|}) \geq 0
\end{equation}
Similarly in the bottom half of the coordinate system with $\xi_2 < 0$ it can be seen:
\begin{equation}
a_1^2\xi_2(\xi_2 - d_2) > a_1^2 \frac{\xi_1}{d_1}d_2(\frac{\xi_1}{d_1}d_2 - d_2) = a_1^2 \xi_2^2(1+\frac{d_1}{\| \xi_1\|}) \geq 0
\end{equation}
And on the separating axis with $\xi_2 = 0$, the therm results as:
\begin{equation}
a_1^2\xi_2(\xi_2 - d_2) = 0  \quad \text{if}
\end{equation}

Moreover, by only observing the left hand side of the ellipse ($\xi_1 < 0$) and having the attractor on the right hand side ($d_1 > 0$), the second part of the third factor of (\ref{eq:crossProduct}) can be  evaluated as:
\begin{equation}
  a_2^2 \xi_1 (\xi_1 - d_1)  > 0
\end{equation}
Having additionally the condition on the eigenvalues that $ \lambda_n(\tilde \xi) - \lambda_{1}(\tilde \xi) > 0$ the first factor is positive, too. Having only positive factors in (\ref{eq:crossProduct}), the inequality in (\ref{eq:condition_noMimima}) is shown to be true.


When evaluating for points below the centerline as stated in (\ref{eq:condition_noMimima2}), only the second product changes sign:
\begin{equation}
\frac{d_1\xi_2 - d_2 \xi_1}{a_2^2 \xi_1^2 + a_1^2 \xi_2^2} < 0
\end{equation}
The resulting crossproduct in (\ref{eq:crossProduct}) is therefore negative. Finally, it can be concluded that any point above the centerline turns in positive direction, while any point below turns in negative direction. \\
\hfill $\blacksquare$




%%% Local Variables:
%%% mode: latex
%%% TeX-master: "../main"
%%% End:

%%% Local Variables:
%%% mode: latex
%%% TeX-master: "../main"
%%% End:


% appendixes should appear before the acknowledgment.

% \section*{acknowledgment}


% the preferred spelling of the word òacknowledgmentó in america is without an òeó after the ògó. avoid the stilted expression, òone of us (r. b. g.) thanks . . .ó  instead, try òr. b. g. thanksó. put sponsor acknowledgments in the unnumbered footnote on the first page.
%%%%%%%%%%%%%%%%%%%%%%%%%%%%%%%%%%%%%%%%%%%%%%%%%%%%%%%%%%%%%%%%%%%%%%%%%%%%%%%%

% references are important to the reader; therefore, each citation must be complete and correct. if at all possible, references should be commonly available publications.
%\bibliographystyle{plain}
\printbibliography

% \begin{thebibliography}{99}
% \bibitem{c1} g. o. young, òsynthetic structure of industrial plastics (book style with paper title and editor),ó 	in plastics, 2nd ed. vol. 3, j. peters, ed.  new york: mcgraw-hill, 1964, pp. 15ð64.
% \bibitem{c2} w.-k. chen, linear networks and systems (book style).	belmont, ca: wadsworth, 1993, pp. 123ð135.
% \bibitem{c3} h. poor, an introduction to signal detection and estimation.   new york: springer-verlag, 1985, ch. 4.
% \end{thebibliography}

\end{document}

%%% Local Variables:
%%% mode: latex
%%% TeX-master: t
%%% End:
