\subsubsection{Multiple Static Ellipses}
A good performance in a DS with four static obstacles is achieved with the DMM, where most trajectories converge to the global minimum (Fig.~\ref{fig:singleAttractor_severalObstacles_DMM_time0}). Only the elongated ellipse poses a problems for the algorithm as a local minimum is formed on its surface (southwest).\\
The IFD avoids the obstacles in a very natural way, the bifurcation seems to minimize the path around the obstacle. Moreover, several obstacles or narrow pathways pose no problem (Fig.~\ref{fig:singleAttractor_severalObstacles_IFD_time0}).

%The robots reach the attractor much slower, when obstacles are placed around it. \\ -- Simulation
\begin{figure}[tb]\centering
% \begin{subfigure}{.48\columnwidth} %
% \centering
% \includegraphics[width=\textwidth]{fig/singleAttractor_severalObstacles___time0.eps}
% \caption{-}
% \label{fig:singleAttractor_severalObstacles___time0}
% \end{subfigure} %
\begin{subfigure}{.48\columnwidth} %
\centering
\includegraphics[width=\textwidth]{fig/singleAttractor_severalObstacles_DMM_time0.eps}
\caption{DMM}
\label{fig:singleAttractor_severalObstacles_DMM_time0}
\end{subfigure}%
\begin{subfigure}{.48\columnwidth} %
\centering
\includegraphics[width=\textwidth]{fig/singleAttractor_severalObstacles_IFD_time0.eps}
\caption{IFS}
\label{fig:singleAttractor_severalObstacles_IFD_time0}
\end{subfigure}
\caption{Obstacle avoidance of four static obstacle placed in a DS with a single attractor.}
\label{fig:singleAttractor_severalObstacles_time}
\end{figure}

%%% Local Variables:
%%% mode: latex
%%% TeX-master: "../main"
%%% End:
