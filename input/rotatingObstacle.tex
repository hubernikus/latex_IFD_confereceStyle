\subsubsection{Single Rotating Object}
The simulation of a rotating ellipse close to the position of the attractor yields (Fig.~\ref{fig:rotating_circle_fig}), that with DMM many trajectories turn in the mathematically positive way (same sense as the obstacle). Furthermore many trajectories move to a local minimum on the obstacles boundary (east). Apart from that, the convergence of the other trajectories works well. \\
The IFD has also many paths turning in the mathematically positive way, but more paths turn the in the negative sens than with IFD. Likewise, a large avoidance region can be observed south of the obstacle, where a fast rotational velocity is expected. Southwest of the obstacle, where it is rotating away from the DS, a \textif{sucking} effect is observed. The trajectories are pulled back to the obstacle, which seems to be analogous to an incompressible fluid. \\
\begin{figure}[tb]\centering
% \begin{subfigure}{.48\columnwidth} %
% \centering
% \includegraphics[width=\textwidth]{fig/rotating_ellipse___time28.eps}
% \caption{-}
% \label{fig:rotating_ellipse___time0}
% \end{subfigure} %
\begin{subfigure}{.48\columnwidth} %
\centering
\includegraphics[width=\textwidth]{fig/rotating_ellipse_DMM_time28.eps}
\caption{DMM}
\label{fig:rotating_ellipse_DMM_time0}
\end{subfigure}%
\begin{subfigure}{.48\columnwidth} %
\centering
\includegraphics[width=\textwidth]{fig/rotating_ellipse_IFD_time28.eps}
\caption{IFD}
\label{fig:rotating_ellipse_IFD_time0}
\end{subfigure}
\caption{Obstacle avoidance of a rotating ellipse in a DS with a single attractor (to direction and magnitude of the rotation is visualized with the red arrow).}
\label{fig:rotating_circle_fig}
\end{figure}

% \begin{figure}[H]\centering
% \begin{subfigure}{.24\textwidth} %
% \centering
% \includegraphics[width=\textwidth]{fig/rotating_circle_none_fig0}
% \caption{}
% \label{fig:rotating_circle_none_fig0}
% \end{subfigure}\,\, %
% \begin{subfigure}{.24\textwidth} %
% \centering
% \includegraphics[width=\textwidth]{fig/rotating_circle_none_fig1}
% \caption{}
% \label{fig:rotating_circle_none_fig1}
% \end{subfigure}
% \begin{subfigure}{.24\textwidth} %
% \centering
% \includegraphics[width=\textwidth]{fig/rotating_circle_none_fig2}
% \caption{}
% \label{fig:rotating_circle_none_fig2}
% \end{subfigure}
% \begin{subfigure}{.24\textwidth} %
% \centering
% \includegraphics[width=\textwidth]{fig/rotating_circle_none_fig2}
% \caption{}
% \label{fig:rotating_circle_none_fig3}
% \end{subfigure}

% \begin{subfigure}{.24\textwidth} %
% \centering
% \includegraphics[width=\textwidth]{fig/rotating_circle_ellipsoid_fig0}
% \caption{}
% \label{fig:rotating_circle_ellipsoid_fig0}
% \end{subfigure}\,\, %
% \begin{subfigure}{.24\textwidth} %
% \centering
% \includegraphics[width=\textwidth]{fig/rotating_circle_ellipsoid_fig1}
% \caption{}
% \label{fig:rotating_circle_ellipsoid_fig1}
% \end{subfigure}
% \begin{subfigure}{.24\textwidth} %
% \centering
% \includegraphics[width=\textwidth]{fig/rotating_circle_ellipsoid_fig2}
% \caption{}
% \label{fig:rotating_circle_ellipsoid_fig2}
% \end{subfigure}
% \begin{subfigure}{.24\textwidth} %
% \centering
% \includegraphics[width=\textwidth]{fig/rotating_circle_ellipsoid_fig3}
% \caption{}
% \label{fig:rotating_circle_ellipsoid_fig3}
% \end{subfigure}

% \begin{subfigure}{.24\textwidth} %
% \centering
% \includegraphics[width=\textwidth]{fig/rotating_circle_fluid_fig0}
% \caption{}
% \label{fig:rotating_circle_fluid_fig0}
% \end{subfigure}\,\, %
% \begin{subfigure}{.24\textwidth} %
% \centering
% \includegraphics[width=\textwidth]{fig/rotating_circle_fluid_fig1}
% \caption{}
% \label{fig:rotating_circle_fluid_fig1}
% \end{subfigure}
% \begin{subfigure}{.24\textwidth} %
% \centering
% \includegraphics[width=\textwidth]{fig/rotating_circle_fluid_fig2}
% \caption{}
% \label{fig:rotating_circle_fluid_fig2}
% \end{subfigure}
% \begin{subfigure}{.24\textwidth} %
% \centering
% \includegraphics[width=\textwidth]{fig/rotating_circle_fluid_fig3}
% \caption{}
% \label{fig:rotating_circle_fluid_fig3}
% \end{subfigure}

% \begin{subfigure}{.24\textwidth} %
% \centering
% \includegraphics[width=\textwidth]{fig/rotating_circle_rotation_fig0}
% \caption{}
% \label{fig:rotating_circle_rotation_fig0}
% \end{subfigure}\,\, %
% \begin{subfigure}{.24\textwidth} %
% \centering
% \includegraphics[width=\textwidth]{fig/rotating_circle_rotation_fig1}
% \caption{}
% \label{fig:rotating_circle_rotation_fig1}
% \end{subfigure}
% \begin{subfigure}{.24\textwidth} %
% \centering
% \includegraphics[width=\textwidth]{fig/rotating_circle_rotation_fig2}
% \caption{}
% \label{fig:rotating_circle_rotation_fig2}
% \end{subfigure}
% \begin{subfigure}{.24\textwidth} %
% \centering
% \includegraphics[width=\textwidth]{fig/rotating_circle_rotation_fig2}
% \caption{}
% \label{fig:rotating_circle_rotation_fig2}
% \end{subfigure}
% \caption{Object avoidance of a rotating ellipse obstacle by a linear attractor DS with DMM (e - h), IFD (i - l) and LRS (m - p) compared to the dynamical system without any objects (a - d)}
% \label{fig:rotating_circle_fig}
% \end{figure}

%%% Local Variables:
%%% mode: latex
%%% TeX-master: "../main"
%%% End:
