\subsubsection{Quantitative comparison}
For the quantitative comparison, a set of initial conditions is chosen and the simulation is run with either IFD or DMM until convergence or the time limit is reached. \\ %
According to the SRC measure the IFD modifies the DS less (Tab.~\ref{tab:twoMovingRotatingObs} -\ref{tab:fastMovingEllipse}), as it has a lower value of at least 15\%  compared to the DMM. \\
The IFD tends to have more collisions, with up to 13\% of the trajectories colliding, while the DMM never collides with an obstacle (Tab.~\ref{tab:fastMovingEllipse}). \\
The convergence time $T_{conv}$ is in the same range for IFD and DMM (Tab.~\ref{tab:fourStaticObjects}).The path length $D_{conv}$ and the relative convergence energy $\hat E_{conv}$ on the other hand are lower for the IFD during the three simulations. The DMM slightly lags having around 10\%-20\% higher convergence distance and up to 100\% higher relative convergence energy. In general, the energy seems to correlate with the value of the SRC. \\
The computational time for the IFD and the DMM are in the same range and vary only by about 10\%: the IFD has a higher computational time with several obstacles.  \\
The complexity based on the length of the code is comparable for both algorithms with around 160 ($\pm$20) lines.

\begin{table}[h]
\centering
\caption{Simulation metrics of a dynamical system with two moving obstacles with time step $dt = 0.005 s$, maximum iteration $i_{max} = 1200$ (similar to Fig.~\ref{fig:twoMovingRotatingObs_DMM} \& \ref{fig:twoMovingRotatingObs_IFD}).}
\label{tab:twoMovingRotatingObs}
\begin{tabular}{|l|r|r|r|} \hline
 &\multicolumn{1}{c|}{-} &\multicolumn{1}{c|}{DMM} &\multicolumn{1}{c|}{IFD} \\ \hline
SRC $[m2/s2]$ &0.0000 &197.6862 &166.6479 \\ \hline
$N_{pen} \, [\%]$ &0.0 &0.0 &0.0 \\ \hline
$T_{pen} []$ &0 &0 &0 \\ \hline
$T_{conv} [s]$ &5.3050 &5.9900 &5.9900 \\ \hline
$D_{conv} [m]$ &22.1793 &25.7054 &24.0317 \\ \hline
$\hat E_{conv} [J/kg]$ &25.3305 &87.8503 &61.5909\\ \hline
$T_{comp} [ms]$ &0.0000 &0.0113 &0.0116 \\ \hline
\end{tabular}
\end{table}


\begin{table}[h]
\centering
\caption{Simulation metrics of a dynamical system with four static objects  with time step $dt = 0.01 s$, maximum iteration $i_{max} = 1200$ and convergence tolerance $r_\epsilon = 0.1 m$ (similar to Fig.~\ref{fig:singleAttractor_severalObstacles_time}).}
\label{tab:fourStaticObjects}
\begin{tabular}{|l|r|r|r|} \hline
 &\multicolumn{1}{c|}{-} &\multicolumn{1}{c|}{DMM} &\multicolumn{1}{c|}{IFD} \\ \hline
SRC $[m2/s2]$ &0.0000 &129.2555 &99.4898 \\ \hline
$N_{pen} \; [\%]$ &0.0 &0.0 &3.3 \\ \hline
$T_{pen} [-] $ &0 &0 &2 \\ \hline
$T_{conv} [s]$ &4.6250 &9.6500 &9.7750 \\ \hline
$D_{conv} [m]$ &9.6599 &12.1595 &11.6221 \\ \hline
$\hat E_{conv} [J/kg]$ &37.0222 &127.2202 &55.9537 \\ \hline
$T_{comp} [ms]$ &0.0000 &0.0096 &0.0102 \\ \hline
\end{tabular}
\end{table}



\begin{table}[h]
\centering
\caption{Simulation metrics of a dynamical system with a fast moving ellipse (similar to Fig.~\ref{fig:fastMovingEllipse_fig}) with time step $dt = 0.005 s$, maximum iteration $i_{max} = 1200$ and convergence tolerance $r_\epsilon = 0.1 m$ .}
\label{tab:fastMovingEllipse}
\begin{tabular}{|l|r|r|r|} \hline
 &\multicolumn{1}{c|}{-} &\multicolumn{1}{c|}{DMM} &\multicolumn{1}{c|}{IFD} \\ \hline
SRC $[m2/s2]$ &0.00 &665.40 &322.58 \\ \hline
$N_{pen} \, [\%]$ &0.0 &0.0 &13.3\\ \hline
$T_{pen} [-]$ &0 &0 &54 \\ \hline
$T_{conv} [s]$ &5.3050 &5.9900 &5.8150 \\ \hline
$D_{conv} [m]$ &22.2363 &31.9179 &26.5775 \\ \hline
$\hat E_{conv} [J/kg]$ &19.1033 &175.4631 &109.4048 \\ \hline
$T_{comp} [ms]$ &0.0000 &0.0079 &0.0072 \\ \hline
\end{tabular}
\end{table}

% \begin{table}[h]
% \centering
% \caption{Complexity of the algorithm is analyzed as the number of lines of acting code (without comments and backspaces) used to express the obstacle avoidance. }
% \label{tab:computationalComplexity}
% \begin{tabular}{|l|r|} \hline
% & Length of code [\# of lines]\\ \hline
% DMM & 143 \\ \hline
% IFD & 183 \\ \hline
% LRS & 169 \\ \hline
% \end{tabular}
% \end{table}




%%% Local Variables:
%%% mode: latex
%%% TeX-master: "../main"
%%% End:
