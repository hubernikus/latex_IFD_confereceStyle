\subsection{Performance}
\begin{figure} [H]
\centering
\includegraphics[width=0.7\textwidth]{fig/quiverPlot_DS_LS_2x_objects61_ang45.eps}
\caption{Original DS (black) and locally rotated (red) of a linear single attractor with a single elliptic object. }
\label{fig:obj61}
\end{figure}

\begin{figure} [H]
\centering
\includegraphics[width=0.7\textwidth]{fig/quiverPlot_DS_LS_objectPos61_rot0.eps}
\caption{Original DS (black) and locally rotated (red) of a linear single attractor with a single elliptic object. }
\label{fig:obj61}
\end{figure}

\begin{figure} [H]
\centering
\includegraphics[width=0.7\textwidth]{fig/quiverPlot_DS_LS_2x_objects52_ang45.eps}
\caption{Original DS (black) and locally rotated (red) of a linear single attractor with a concave object. The dynamical system modifies within and outside the concave region. }
\label{fig:obj61}
\end{figure}

\begin{figure} [H]
\centering
\includegraphics[width=0.7\textwidth]{fig/quiverPlot_DS_LS_2x_objects52_ang45_close.eps}
\caption{Original DS (black) and locally rotated (red) of a linear single attractor with a concave object. The dynamical system modifies within and outside the concave region.} 
\label{fig:obj61} 
\end{figure}

% \subsubsection{Animations}
% \textbf{localRotation-avoidCicrcle.gif}: Rotate space to avoid circle that is lying in the middle.\\
% \textbf{localRotation-avoidEllipses-startingOutside.gif}: Avoid the concave region created by two circles. \\
% \textbf{localRotation-avoidEllipses-startingInside.gif}: Exit the concave region created by two ellipses, when starting inside.. \\