\appendix
\subsection{Proof of Theorem 1} \label{sec:proof1}
 %\textbf{Impenetrability}
%Any coordinate transformation $\mathcal{E} \to \mathcal{C}:  \mathbb{R}^d \to \mathbb{R}^d$ which transforms a convex region in $\mathcal{X}_1^d \subset \mathcal{E}$ to a convex region in $\mathcal{X}_2^d \subset \mathcal{C}$, ensures that any vector starting and ending in the convex region $\dot \xi_{\mathcal{E}} \in \mathcal{X}_1$ will be contained in the convex region $\dot \xi_{\mathcal{C}} \in \mathcal{X}_2$, vise versa any vector outside the starting and ending outside will start and end outside this region.

%The linear transformation with the stretching matrix $A$ and rotation matrix $R$, as well as the translation with $\xi_0$ keep the convex regions intact. As the velocity vector starts and stays outside the concave circle and will therefore stay that way for the transformation back to the ellipse space, while impenetrability is  given. \hfill $\blacksquare$ \vspace{1em}

% The surface of the ellipse can be described as:
% \begin{equation}
%  \frac {{}_{e}\xi_1 ^ 2}{a^2} + \frac{{}_e\xi_2^2}{b^2} = 1
% \end{equation}
Without any loss of generalization, the proof is done with the obstacle placed at the origin ($\xi \tilde \xi$) and the main axis of the ellipse aligned with the coordinate system. The subscript ${}_e (\cdot)$ declares that a variable is given in the ellipse reference frame (ERF).
With the Jacobian of the level function of an ellipse described in (\ref{eq:levelEllipse}), a normal ${}_e \vec n_e (\xi)$ on the ellipse can be found:
\begin{equation}
  {}_e \vec{n}_{e} (\xi) =
  \frac{1}{\sqrt{a_2^4 \cdot {}_e\xi_1^2 + a_1^4 \cdot{}_e \xi_2^2}}
  \begin{bmatrix}
    a_2^2 \cdot {}_e\xi_1 \\
    a_1^2 \cdot {}_e\xi_2
  \end{bmatrix}
\end{equation}

The unit tangent on the ellipse ${}_e\vec e_{e}$ is perpendicular to the normal, and can be written as:
\begin{equation}
  {}_e \vec{e}_{e} (\xi)=
  \frac{1}{\sqrt{a_2^4 \cdot {}_e\xi_1^2 + a_1^4 \cdot{}_e \xi_2^2}}
  \begin{bmatrix}
    a_1^2 \cdot {}_e\xi_2 \\
    - a_2^2 \cdot {}_e\xi_1
  \end{bmatrix}
\end{equation}

Before the modulation is applied, the whole coordinate system is transformed, such that the ellipse lies on a unit circle, which is referred as circle reference frame (CRF) with the subscript ${}_c(\cdot)$:
\begin{equation}
  %{}_c \xi = A^{-1} {}_e \xi \quad \rightarrow \quad
  {}_e \xi_1 (\xi) = a_1 \cdot {}_c\xi_1 \, , \;\; {}_e \xi_2 = a_2 \cdot {}_c\xi_2 \label{eq:ellipseCircleTrafo}
\end{equation}
%For a unit circle, the two axis have the same value, and are given by $r = a = b = 1$. \\
% \begin{equation}
%  \xi_1 ^ 2 + \xi_2^2 = 1
% \end{equation}
The normal ${}_c \vec n _{c} (\xi)$ of the unit circle in CRF can be evaluated similarly with the level function (\ref{eq:levelCircle})
\begin{equation}
  {}_c \vec{n}_{c} (\xi) =
  \frac{1}{\sqrt{ {}_c\xi_1^2 + {}_c \xi_2^2}}
  \begin{bmatrix}
     {}_c\xi_1 \\
     {}_c\xi_2
  \end{bmatrix}
\end{equation}

and the tangent
\begin{equation}
  {}_c \vec{e}_{c} (\xi) =
  \frac{1}{\sqrt{ {}_c\xi_1^2 + {}_c \xi_2^2}}
  \begin{bmatrix}
     {}_c\xi_2 \\
     - {}_c\xi_1
  \end{bmatrix}
\end{equation}

On the obstacle surface, the velocity needs to be solely in tangential direction. While this was shown for the circle, the proof has to be extended to the ellipse. For this, the tangent on the unit circle needs to be equal to the tangent of the ellipse, too. To evaluate this, the tangent on the unit circle in CRF is transformed to the ERF:
\begin{equation}
  {}_e \vec{e}_{c} (\xi) = \frac{1}{k_{ec}}\begin{bmatrix} a_1 & 0 \\ 0 & a_2\end{bmatrix} \cdot {}_c \vec{e}_{c}(\xi)
    = \frac{1}{\sqrt{a^2 \cdot {}_c\xi_2^2 + a_2^2 {}_c\xi_1^2}}
  \begin{bmatrix}
       a_1 \cdot {}_c\xi_2 \\
     - a_2 \cdot {}_c\xi_1
   \end{bmatrix}
 \end{equation}
 with $k_{ec}$ begin the normalization factor. \\
Furthermore, the coordinate transformation from (\ref{eq:ellipseCircleTrafo}) leads to:
\begin{equation}
  {}_e \vec{e}_{c} (\xi) =
  \frac{1}{\sqrt{a_2^4 \cdot {}_e\xi_1^2 + a^4 \cdot{}_e \xi_2^2}}
  \begin{bmatrix}
    a_1^2 \cdot {}_e\xi_2 \\
    - a_2^2 \cdot {}_e\xi_1
  \end{bmatrix}
   =
    {}_e \vec{e}_{e} (\xi)
\end{equation}
which is equal to the tangent on the ellipse in ellipse space. Projecting the velocity solely on the tangent in ellipse space is ensured and therefore impenetrability is guaranteed. \hfill $\blacksquare$ \\

It is important to notice, that the normal on the circle transformed to the ERF ${}_e \vec n_{c} $ is not normal to the ellipse:
\begin{equation}
  {}_e\vec n_{c} (\xi)
  =
  \frac{1}{a_1 a_2\sqrt{ {}_e\xi_1^2 + {}_e \xi_2^2}}
  \begin{bmatrix}
     a_1 a_2 \cdot{}_e\xi_1 \\
     a_1 a_2 \cdot {}_e\xi_2
   \end{bmatrix}
  \neq
  {}_e\vec n_{e} (\xi)
\end{equation}
%As a result of this, the normal and tangent of the CRF are not orthogonal in ERF, but linearly independent. Zero velocity in direction of the normal on the ellipse can still be ensured, because the normal on the ellipse ${}_e \vec n_{e} (\xi)$ is a linear combination of the normal on the circle in ERF (${}_e \vec n_{c} (\xi)$, ${}_e \vec e_{e}(\xi)$). As there is no velocity in direction of ${}_e \vec n_{c} (\xi) $ it is a trivial normal combination resulting in ${}_e \vec n_{e} (\xi)= 0$.


\subsection{Proof of Theorem 2} \label{sec:proof2}
Given is the base matrix $\hat{D} = [\hat n (\xi) \; e_1(\xi) ]$  with $e_1(\xi) $ the tangent vector on the surface and $\hat n$ a vector linearly independent, but not necessarily the normal to the tangent..

For the two dimensional case, a $\hat{ \vec n}$  which is inclined at an angle of $\phi \in ]-\frac{\pi}{2}, \frac{\pi}{2}[$ towards the normal on the surface $\vec n$. Such that:
\begin{equation}
{\vec n} \times \hat{\vec n} = \sin \phi
\end{equation}
Furthermore, $\hat{\ve n}$ can be expressed as a linear combination of the normal $\vec n$ and the tangent $\vec e_1$:
\begin{equation}
\hat{ \vec n} (\xi) = \cos \phi \cdot \vec n (\xi) + \sin \phi \cdot \vec e_1 (\xi)
\end{equation}
At any position the DS can be represented as a linear combination of the bases matrix $\hat D$:
\begin{equation}
  f(\xi) = f_1(\xi)\cdot \hat{\vec n} + f_2(\xi) \cdot \vec{e}_1
\end{equation}
where $f_1(\xi)$ and $f_2(\xi)$ are the linear factors.
For any point on the surface of the obstacle ($\Gamma(\xi) = 1$), the eigenvalues given in (\ref{eq:eigenValues}) are zero for the pseudo normal component, but have a value of $\lambda(\xi)_1$ in tangential direction. The modulated DS therefore results as:
\begin{equation}
  \dot \xi = 0 \cdot \hat{\vec n} + \lambda_1(\xi) f_2(\xi) \cdot \vec{e}_1
\end{equation}
Impenetrability can now be checked with the Neuman boundary condition:
\begin{align}
  \dot{\vec \xi} \cdot \vec n (\xi) &=  \dot{\vec \xi} \cdot \left(\frac{1}{\cos \phi} \hat{\vec n}(\xi) - \frac{\sin \phi}{\cos \phi} \vec{e}_1 (\xi) \right) \\
  & = \left| \dot{\vec \xi}  \right| \cdot \left(\frac{\cos(\frac{\pi}{2} - \phi)
}{\cos \phi} - \frac{\sin \phi}{\cos \phi}\right) = 0
\end{align}
Impenetrability is therefore given. \hfill $\blacksquare$



\subsection{Proof of Theorem 3} \label{sec:proof3}
Without any loss of generalization, the 2-dimensional ellipse is placed at the origin, $\xi^o = [0 \; \; 0]^T$, therefore we have $\xi = \tilde \xi$. Furthermore, the main axes $a_1$ and $a_2$ are aligned with $\xi_1$ and $\xi_2$, respectively. The linear attractor is placed at $\xi^a = \begin{bmatrix} d_1 & d_2 \end{bmatrix}$ with $d_1 \in \mathbb{R}^\plus$ and $d_2 \in \mathbb{R}$. The DS for such a case is given as $f(\xi) = k_f(\xi) \cdot \begin{bmatrix} -(\xi_1-d_1) & -(\xi_2-d_2)\end{bmatrix}^T$, where $k_f(\xi) \in \mathbb{R}^+$ is a scalar, stretching factor of the DS.
The modulation matrix $\hat {\vec{M}}$ as described in Sec.~\ref{sec:IFD} is given as:
\begin{align}
  \hat M ( \xi) &= \hat{E}(\xi) D(\xi) \hat{E}(\xi) %\\
\end{align}
With following components:
\begin{equation}
  D(\xi) =
  \begin{bmatrix}
    \lambda_n(\tilde \xi) & 0 \\
    0 & \lambda_1(\tilde \xi)
  \end{bmatrix}
  \qquad
  \hat{E} (\xi) =
  \begin{bmatrix}
    \hat{\vec n } & \vec e_1
  \end{bmatrix}
\end{equation}
where $\hat{\vec n} = \begin{bmatrix}$ \xi_1 & \xi_2 \end{bmatrix}$ and $\vec e_1 = \begin{bmatrix} a_1^2 \xi_2 & - a_2^2 \xi_1 \end{bmatrix}$.\\
To show that the modulates DS goes towards the origin, it needs to be shown that any trajectory above the saddle point trajectory $\mathcal{X}^s$ as given in (\ref{eq:centerLine}) turns away from the ellipse center: a rotation in positive direction.  Any point below $\mathcal{X}^s$ needs to turn in the opposite direction.
This condition is given as:
\begin{align}
  &(-\hat{\vec{n}}) \times \dot \xi > 0 \quad \text{if} \;\; \xi_2 > \frac{d_2 \xi_1}{d_1}  \label{eq:condition_noMimima} \\
  &(-\hat{\vec n}) \times \dot \xi =  0  \quad \text{if} \;\; \xi_2 = \frac{d_2 \xi_1}{d_1}   \label{eq:condition_noMinima0} \\
  &(-\hat{\vec{n}}) \times \dot \xi < 0   \quad \text{if} \;\; \xi_2 < \frac{d_2 \xi_1}{d_1}   \label{eq:condition_noMimima2}
  \end{align}

The modulated DS is given by:
\begin{align}
  \dot{\xi} &= \hat{M}(\xi) \cdot f(\xi)
  % \begin{pmatrix}
  %    \lambda_n(\tilde \xi) (-a_2^2 \xi_1^3 + a_2^2 d \xi_1^2 - a_1^2 \xi_1 \xi_2^2) +   \lambda_{e1}(\tilde \xi) a_1^2 d \xi_2^2 \\
  %    \lambda_n(\tilde \xi) (-a_2^2 \xi_1^2 \xi_2 + a_2^2 d \xi_1 \xi_2 - a_1^2 \xi_2^3) +   \lambda_{e1}(\tilde \xi) a_2^2 d \xi_1 \xi_2
  %\end{pmatrix}
\end{align}

The cross product is evaluated as:
% \begin{equation} \label{eq:crossProduct}
%   \begin{split}
%    f(\xi) \times \dot \xi =&  k_f^2(\xi) \left( \lambda_n(\tilde \xi) - \lambda_{1}(\tilde \xi) \right) \frac{d_1\xi_2 - d_2 \xi_1}{a_2^2 \xi_1^2 + a_1^2 \xi_2^2} \cdot \\
%   & \left(a_1^2\xi_2(\xi_2 - d_2) + a_2^2 \xi_1 (\xi_1 - d_1)  \right)
%   % \left( \lambda_n(\tilde \xi) - \lambda_{e1}(\tilde \xi) \right) \left( d a_2^2 \xi_1^2 \xi_2 - a_2^2 d^2 \xi_1 \xi_2 + a_1^2 d \xi_2^2 \right)
%   \end{split}
% \end{equation}
\begin{equation}
  (- \hat{\vec n}) \times \dot \xi  =
  k_f^2(\xi)  \lambda_{1}(\tilde \xi) \cdot \left( d_1\xi_2 - d_2 \xi_1\right) \label{eq:crossProduct}
\end{equation}

With condition condition (\ref{eq:condition_noMimima}) and with the condition for the eigenvalue $\lambda_1(\xi) > 0 $ it can easily be seen, that the cross product (\ref{eq:crossProduct}) is positive above the saddle point trajectory. Which can be interpreted as a rotation of the modulated DS in positive direction with respect to the inverse of the pseudo normal. The DS is therefore guided around the obstacle, until it reaches the attractor $\mathcal{X}^a$, which lies on the saddle point trajectory $\mathcal{X}^s$. On this trajectory, the cross product (\ref{eq:crossProduct}) is zero, either because there is no change in direction or the DS has reached the saddle point, in which case it has no velocity.
Similarly, the cross product for point below the saddle point described in (\ref{eq:crossProduct}) is evaluated as negative. The DS  is therefore rotated in negative direction. Conditions (\ref{eq:condition_noMimima}, \ref{eq:condition_noMinima0}, \ref{eq:condition_noMimima2}) are proofed true. \\
\hfill $\blacksquare$




% First we try to show, that vector $f(\xi)$ above the centerline is rotated clockwise as described in (\ref{eq:condition_noMimima}).\\
% The third factor in (\ref{eq:crossProduct}) can be evaluated with (\ref{eq:condition_noMimima})  as:
% \begin{equation}
%  \frac{d_1\xi_2 - d_2 \xi_1}{a_2^2 \xi_1^2 + a_1^2 \xi_2^2} > 0
% \end{equation}
% The first summand of the fourth factor in (\ref{eq:crossProduct}) is observed in the three different regions. \\
% In the top half of the coordinate system with $\xi_2 > 0$, using (\ref{eq:condition_noMimima}) it follows:
% \begin{equation}
%   a_1^2\xi_2(\xi_2 - d_2) > a_1^2\xi_2(\xi_2 - \xi_2 \frac{d_1}{\xi_1}) = a_1^2 \xi_2^2(1+\frac{d_1}{\| \xi_1\|}) \geq 0
% \end{equation}
% Similarly in the bottom half of the coordinate system with $\xi_2 < 0$ it can be seen:
% \begin{equation}
% a_1^2\xi_2(\xi_2 - d_2) > a_1^2 \frac{\xi_1}{d_1}d_2(\frac{\xi_1}{d_1}d_2 - d_2) = a_1^2 \xi_2^2(1+\frac{d_1}{\| \xi_1\|}) \geq 0
% \end{equation}
% And on the separating axis with $\xi_2 = 0$, the therm results as:
% \begin{equation}
% a_1^2\xi_2(\xi_2 - d_2) = 0  \quad \text{if}
% \end{equation}

% Finally observing the second summand of the fourth factor in (\ref{eq:crossProduct}), using (\ref{eq:condition_noMimima}) it can be shown that:
% \begin{equation}
%   a_2^2 \xi_1 (\xi_1 - d_1)  > 0
% \end{equation}
% Having additionally the condition on the eigenvalues that $ \lambda_n(\tilde \xi) - \lambda_{1}(\tilde \xi) > 0$ the second factor is positive, too. Having only positive factors in (\ref{eq:crossProduct}), the inequality in (\ref{eq:condition_noMimima}) is true.

% When evaluating for points below the saddle point trajectory as stated in (\ref{eq:condition_noMimima2}), only the second product changes the sign:
% \begin{equation}
% \frac{d_1\xi_2 - d_2 \xi_1}{a_2^2 \xi_1^2 + a_1^2 \xi_2^2} < 0
% \end{equation}
% The resulting cross product in (\ref{eq:crossProduct}) is concluded to be negative. Therefore, it is shown that any point above the centerline turns in positive direction, while any point below turns in negative direction. \\
% \hfill $\blacksquare$

% [Comment - TODO: \\
% - This should actually be evaluated for $-\hat{\vec n} (\xi) \times \dot \xi >< 0$, which should be very similar, but it is more relevant. ]

%%% Local Variables:
%%% mode: latex
%%% TeX-master: "../main"
%%% End:
