\section{Comparison Algorithms} \label{sec:comparison}
\subsection{Metrics}
% & $\ddot{\bar \xi} \, [m2/s2]$ & $N_{pen} \, [\%]$ & $T_{pen} []$ & $T_{conv} [s]$ & $D_{conv} [m]$ & $\hat E_{conv} [J/kg]$ & $T_{comp} [ms]$ \\ \hline
Various metrics are defined in order to quantify the performance of the different obstacle avoidance algorithms based on DS.

\subsubsection{Square of the Relative Change of the DS}
The square of the relative change ($SRC$) gives information on how much the modulated DS $\dot{\bar \xi}$ changed compared to the original DS $\dot \xi$. It is evaluated at several points in space, before it is summed over dimensions $j$ and the samples $i$:
\begin{equation}
SRC = \sum_i \sum_j \left( \dot{\hat{\xi}}_j(\xi) - \dot{\xi}_j(\xi) \right)^2
\end{equation}
In general, the original DS should be modulated as little as possible, while avoiding the obstacle without collision. A low $SRC$ is therefore preferred.

\subsubsection{Penetration Time and Occurrence} This metric observes trajectories based on different starting points during a simulation. The penetration occurrence $N_{pene}$ indicates the percentage of trajectories which collide with the object's boundary; the penetration time $T_{pen}$ shows for how many time steps a penetrating trajectory stayed inside an object's boundary on average. (It is 0 if no penetration occurred.)

\subsubsection{Convergence Time, Distance and Relative Energy}
This metric is used in linear and stable DS with a single attractor, but could be extended to limit cycles or DS with several attractors. Convergence is reached when a trajectory is within $r_\epsilon$ of the stable attractor for several consecutive time steps. The convergence time $T_{conv}$ is the simulation time when the robot reaches convergence and the convergence distance is the sum of the discrete distance steps of its trajectory until convergence is reached:
\begin{equation}
D_{conv} = \sum_{i=1}^{i_{conv}} \left| \xi_i -\xi_{i-1} \right|
\end{equation}
where $i_{conv}$ is the simulation step at convergence. \\
The convergence relative energy is the energy it takes to reach the convergence region. It is evaluated as the discrete acceleration multiplied with the distance integrated over time. Where the acceleration is the numerical derivative of the velocity:
\begin{equation}
\hat E_{conv} = \sum_i \ddot{\xi}(\xi_i) \cdot \Delta \xi_i = \sum_i \frac{\dot{\xi}(\xi_i) - \dot{\xi}(\xi_{i-1})}{\Delta t} \cdot \left( \xi_i - \xi_{i-1} \right)
\end{equation}

\subsubsection{Computational Cost, Memory usage and Complexity}
An important aspect of an online algorithm is its computational cost. While it is hard to compare the different operations and algorithms theoretically, the computational time $T_{comp} [ms]$ can easily be measured during the execution. The computational time is expected to increase with increasing computational cost. \\
Memory usage of the algorithm is not observed, but it is expected to be low for all of them, because the DS are evaluated with functions and no big data sets are stored. \\
Moreover, the computational complexity is simplified by the number of code lines in the implementation of the algorithms.


\subsection{Results}
Further simulations are executed to compare the algorithms' performance both qualitatively and quantitatively in this section.

\subsubsection{Multiple Static Ellipses}
A good performance in a DS with four static obstacles is achieved with the DMM, where most trajectories converge to the global minimum (Fig.~\ref{fig:singleAttractor_severalObstacles_DMM_time0}). Only the elongated ellipse poses a problems for the algorithm as a local minimum is formed on its surface (southwest).\\
The IFD avoids the obstacles in a very natural way, the bifurcation seems to minimize the path around the obstacle. Moreover, several obstacles or narrow pathways pose no problem (Fig.~\ref{fig:singleAttractor_severalObstacles_IFD_time0}).

%The robots reach the attractor much slower, when obstacles are placed around it. \\ -- Simulation
\begin{figure}[tb]\centering
% \begin{subfigure}{.48\columnwidth} %
% \centering
% \includegraphics[width=\textwidth]{fig/singleAttractor_severalObstacles___time0.eps}
% \caption{-}
% \label{fig:singleAttractor_severalObstacles___time0}
% \end{subfigure} %
\begin{subfigure}{.48\columnwidth} %
\centering
\includegraphics[width=\textwidth]{fig/singleAttractor_severalObstacles_DMM_time0.eps}
\caption{DMM}
\label{fig:singleAttractor_severalObstacles_DMM_time0}
\end{subfigure}%
\begin{subfigure}{.48\columnwidth} %
\centering
\includegraphics[width=\textwidth]{fig/singleAttractor_severalObstacles_IFD_time0.eps}
\caption{IFS}
\label{fig:singleAttractor_severalObstacles_IFD_time0}
\end{subfigure}
\caption{Obstacle avoidance of four static obstacle placed in a DS with a single attractor.}
\label{fig:singleAttractor_severalObstacles_time}
\end{figure}

%%% Local Variables:
%%% mode: latex
%%% TeX-master: "../main"
%%% End:


\subsubsection{Fast Translational Moving Object}
When encountering a fast translational moving obstacle, the IFD avoids the obstacle with splitting the linear DS close to the trajectory which points through the center of mass of the obstacle (Fig.~\ref{fig:fastMovingEllipse_IFD_time13}). Furthermore,  trajectories passing in front of the moving obstacle have a larger safety margin, while the  DS in the wake of the velocity is less disturbed. \\
With the DMM, a large part of the trajectories which start behind the obstacle, as seen from the attractor, converge to a local minimum on the ellipse, instead of converging to the desired attractor (Fig.~\ref{fig:fastMovingEllipse_DMM_time13}).

\begin{figure}[tb]\centering
% \begin{subfigure}{.48\columnwidth} %
% \centering
% \includegraphics[width=\textwidth]{fig/fastMovingEllipse___time13.eps}
% \caption{-}
% \label{fig:fastMovingEllipse___time13}
% \end{subfigure} %
\begin{subfigure}{.48\columnwidth} %
\centering
\includegraphics[width=\textwidth]{fig/fastMovingEllipse_DMM_time13.eps}
\caption{DMM}
\label{fig:fastMovingEllipse_DMM_time13}
\end{subfigure}%
\begin{subfigure}{.48\columnwidth} %
\centering
\includegraphics[width=\textwidth]{fig/fastMovingEllipse_IFD_time13.eps}
\caption{IFD}
\label{fig:fastMovingEllipse_IFD_time13}
\end{subfigure}
\caption{Obstacle avoidance of a fast translational moving obstacle placed in a DS with a single attractor.}
\label{fig:fastMovingEllipse_fig}
\end{figure}



% \begin{figure}[H]\centering
% \begin{subfigure}{.24\textwidth} %
% \centering
% \includegraphics[width=\textwidth]{fig/fastMovingEllipse_none_fig0}
% \caption{}
% \label{fig:fastMovingEllipse_none_fig0}
% \end{subfigure}\,\, %
% \begin{subfigure}{.24\textwidth} %
% \centering
% \includegraphics[width=\textwidth]{fig/fastMovingEllipse_none_fig1}
% \caption{}
% \label{fig:fastMovingEllipse_none_fig1}
% \end{subfigure}
% \begin{subfigure}{.24\textwidth} %
% \centering
% \includegraphics[width=\textwidth]{fig/fastMovingEllipse_none_fig2}
% \caption{}
% \label{fig:fastMovingEllipse_none_fig2}
% \end{subfigure}
% \begin{subfigure}{.24\textwidth} %
% \centering
% \includegraphics[width=\textwidth]{fig/rotating_circle_none_fig3} % other one converged to fast
% \caption{}
% \label{fig:fastMovingEllipse_none_fig2}
% \end{subfigure}

% \begin{subfigure}{.24\textwidth} %
% \centering
% \includegraphics[width=\textwidth]{fig/fastMovingEllipse_ellipsoid_fig0}
% \caption{}
% \label{fig:fastMovingEllipse_ellipsoid_fig0}
% \end{subfigure}\,\, %
% \begin{subfigure}{.24\textwidth} %
% \centering
% \includegraphics[width=\textwidth]{fig/fastMovingEllipse_ellipsoid_fig1}
% \caption{}
% \label{fig:fastMovingEllipse_ellipsoid_fig1}
% \end{subfigure}
% \begin{subfigure}{.24\textwidth} %
% \centering
% \includegraphics[width=\textwidth]{fig/fastMovingEllipse_ellipsoid_fig2}
% \caption{}
% \label{fig:fastMovingEllipse_ellipsoid_fig2}
% \end{subfigure}
% \begin{subfigure}{.24\textwidth} %
% \centering
% \includegraphics[width=\textwidth]{fig/fastMovingEllipse_ellipsoid_fig3}
% \caption{}
% \label{fig:fastMovingEllipse_ellipsoid_fig3}
% \end{subfigure}

% \begin{subfigure}{.24\textwidth} %
% \centering
% \includegraphics[width=\textwidth]{fig/fastMovingEllipse_fluid_fig0}
% \caption{}
% \label{fig:fastMovingEllipse_fluid_fig0}
% \end{subfigure}\,\, %
% \begin{subfigure}{.24\textwidth} %
% \centering
% \includegraphics[width=\textwidth]{fig/fastMovingEllipse_fluid_fig1}
% \caption{}
% \label{fig:fastMovingEllipse_fluid_fig1}
% \end{subfigure}
% \begin{subfigure}{.24\textwidth} %
% \centering
% \includegraphics[width=\textwidth]{fig/fastMovingEllipse_fluid_fig2}
% \caption{}
% \label{fig:fastMovingEllipse_fluid_fig2}
% \end{subfigure}
% \begin{subfigure}{.24\textwidth} %
% \centering
% \includegraphics[width=\textwidth]{fig/fastMovingEllipse_fluid_fig3}
% \caption{}
% \label{fig:fastMovingEllipse_fluid_fig3}
% \end{subfigure}

% \begin{subfigure}{.24\textwidth} %
% \centering
% \includegraphics[width=\textwidth]{fig/fastMovingEllipse_rotation_fig0}
% \caption{}
% \label{fig:fastMovingEllipse_rotation_fig0}
% \end{subfigure}\,\, %
% \begin{subfigure}{.24\textwidth} %
% \centering
% \includegraphics[width=\textwidth]{fig/fastMovingEllipse_rotation_fig1}
% \caption{}
% \label{fig:fastMovingEllipse_rotation_fig1}
% \end{subfigure}
% \begin{subfigure}{.24\textwidth} %
% \centering
% \includegraphics[width=\textwidth]{fig/fastMovingEllipse_rotation_fig2}
% \caption{}
% \label{fig:fastMovingEllipse_rotation_fig2}
% \end{subfigure}
% \begin{subfigure}{.24\textwidth} %
% \centering
% \includegraphics[width=\textwidth]{fig/fastMovingEllipse_rotation_fig3}
% \caption{}
% \label{fig:fastMovingEllipse_rotation_fig3}
% \end{subfigure}
% \caption{Object avoidance one fast moving ellipse with a linear attractor DS with DMM (e - h), IFD (i - l) and LRS (m - p) compared to the dynamical system without any objects (a - d)}
% \label{fig:fastMovingEllipse_fig}
% \end{figure}

%%% Local Variables:
%%% mode: latex
%%% TeX-master: "../main"
%%% End:


\subsubsection{Object moving close to origin.}
Fig.~\ref{fig:leftMoving_circle_fig} shows the behavior of the two algorithms, when a circular obstacle is moving close to the attractor position. For the IFD, the global minimum is shifted slightly away in direction opposite to the direction of the obstacle (southwest), this seems to be a good strategy to get a larger safety margin when the obstacle passes. On the other hand, it is not necessary to move, because the obstacle is not going to hit the attractor.\\
The global minimum with the DMM is also shifted to ensure no collision with the approaching obstacle. It is shifted in direction southeast, which is a less optimal option. Additionally, it is shifted less than with the IFD.

\begin{figure}[tb]\centering
% \begin{subfigure}{.48\columnwidth} %
% \centering
% \includegraphics[width=\textwidth]{fig/leftMoving_circle___time20.eps}
% \caption{-}
% \label{fig:leftMoving_circle___time2}
% \end{subfigure} %

\begin{subfigure}{.48\columnwidth} %
\centering
\includegraphics[width=\textwidth]{fig/leftMoving_circle_DMM_time20.eps}
\caption{DMM}
\label{fig:leftMoving_circle_DMM_time2}
\end{subfigure}%
\begin{subfigure}{.48\columnwidth} %
\centering
\includegraphics[width=\textwidth]{fig/leftMoving_circle_IFD_time20.eps}
\caption{IFD}
\label{fig:leftMoving_circle_IFD_time2}
\end{subfigure}
\caption{Obstacle avoidance when moving close to the origin in a DS with a single attractor.}
\label{fig:leftMoving_circle_fig}
\end{figure}

% \begin{figure}[H]\centering
% \begin{subfigure}{.32\textwidth} %
% \centering
% \includegraphics[width=\textwidth]{fig/leftMoving_circle_ellipsoid_fig2}
% \caption{}
% \label{fig:leftMoving_circle_ellipsoid_fig2}
% \end{subfigure}\,\, %
% \begin{subfigure}{.32\textwidth} %
% \centering
% \includegraphics[width=\textwidth]{fig/leftMoving_circle_ellipsoid_fig3}
% \caption{}
% \label{fig:leftMoving_circle_ellipsoid_fig3}
% \end{subfigure}
% \begin{subfigure}{.32\textwidth} %
% \centering
% \includegraphics[width=\textwidth]{fig/leftMoving_circle_ellipsoid_fig4}
% \caption{}
% \label{fig:leftMoving_circle_ellipsoid_fig4}
% \end{subfigure}


% \begin{subfigure}{.32\textwidth} %
% \centering
% \includegraphics[width=\textwidth]{fig/leftMoving_circle_fluid_fig2}
% \caption{}
% \label{fig:leftMoving_circle_fluid_fig2}
% \end{subfigure}\,\, %
% \begin{subfigure}{.32\textwidth} %
% \centering
% \includegraphics[width=\textwidth]{fig/leftMoving_circle_fluid_fig3}
% \caption{}
% \label{fig:leftMoving_circle_fluid_fig3}
% \end{subfigure}
% \begin{subfigure}{.32\textwidth} %
% \centering
% \includegraphics[width=\textwidth]{fig/leftMoving_circle_fluid_fig4}
% \caption{}
% \label{fig:leftMoving_circle_fluid_fig4}
% \end{subfigure}

% \begin{subfigure}{.32\textwidth} %
% \centering
% \includegraphics[width=\textwidth]{fig/leftMoving_circle_rotation_fig2}
% \caption{}
% \label{fig:leftMoving_circle_rotation_fig2}
% \end{subfigure}\,\, %
% \begin{subfigure}{.32\textwidth} %
% \centering
% \includegraphics[width=\textwidth]{fig/leftMoving_circle_rotation_fig3}
% \caption{}
% \label{fig:leftMoving_circle_rotation_fig3}
% \end{subfigure}
% \begin{subfigure}{.32\textwidth} %
% \centering
% \includegraphics[width=\textwidth]{fig/leftMoving_circle_rotation_fig4}
% \caption{}
% \label{fig:leftMoving_circle_rotation_fig4}
% \end{subfigure}
% \caption{Object avoidance of one circle which only starts moving after the solution trajectories have almost converged to the attractor with  DMM(a - c), IFD (d - f) and LRS (g - i)}
% \label{fig:leftMoving_circle_fig}
% \end{figure}

%%% Local Variables:
%%% mode: latex
%%% TeX-master:
%%% "../main"
%%% End:


\subsubsection{Single Rotating Object}
The simulation of a rotating ellipse close to the position of the attractor yields (Fig.~\ref{fig:rotating_circle_fig}), that with DMM many trajectories turn in the mathematically positive way (same sense as the obstacle). Furthermore many trajectories move to a local minimum on the obstacles boundary (east). Apart from that, the convergence of the other trajectories works well. \\
The IFD has also many paths turning in the mathematically positive way, but more paths turn the in the negative sens than with IFD. Likewise, a large avoidance region can be observed south of the obstacle, where a fast rotational velocity is expected. Southwest of the obstacle, where it is rotating away from the DS, a \textif{sucking} effect is observed. The trajectories are pulled back to the obstacle, which seems to be analogous to an incompressible fluid. \\
\begin{figure}[tb]\centering
% \begin{subfigure}{.48\columnwidth} %
% \centering
% \includegraphics[width=\textwidth]{fig/rotating_ellipse___time28.eps}
% \caption{-}
% \label{fig:rotating_ellipse___time0}
% \end{subfigure} %
\begin{subfigure}{.48\columnwidth} %
\centering
\includegraphics[width=\textwidth]{fig/rotating_ellipse_DMM_time28.eps}
\caption{DMM}
\label{fig:rotating_ellipse_DMM_time0}
\end{subfigure}%
\begin{subfigure}{.48\columnwidth} %
\centering
\includegraphics[width=\textwidth]{fig/rotating_ellipse_IFD_time28.eps}
\caption{IFD}
\label{fig:rotating_ellipse_IFD_time0}
\end{subfigure}
\caption{Obstacle avoidance of a rotating ellipse in a DS with a single attractor (to direction and magnitude of the rotation is visualized with the red arrow).}
\label{fig:rotating_circle_fig}
\end{figure}

% \begin{figure}[H]\centering
% \begin{subfigure}{.24\textwidth} %
% \centering
% \includegraphics[width=\textwidth]{fig/rotating_circle_none_fig0}
% \caption{}
% \label{fig:rotating_circle_none_fig0}
% \end{subfigure}\,\, %
% \begin{subfigure}{.24\textwidth} %
% \centering
% \includegraphics[width=\textwidth]{fig/rotating_circle_none_fig1}
% \caption{}
% \label{fig:rotating_circle_none_fig1}
% \end{subfigure}
% \begin{subfigure}{.24\textwidth} %
% \centering
% \includegraphics[width=\textwidth]{fig/rotating_circle_none_fig2}
% \caption{}
% \label{fig:rotating_circle_none_fig2}
% \end{subfigure}
% \begin{subfigure}{.24\textwidth} %
% \centering
% \includegraphics[width=\textwidth]{fig/rotating_circle_none_fig2}
% \caption{}
% \label{fig:rotating_circle_none_fig3}
% \end{subfigure}

% \begin{subfigure}{.24\textwidth} %
% \centering
% \includegraphics[width=\textwidth]{fig/rotating_circle_ellipsoid_fig0}
% \caption{}
% \label{fig:rotating_circle_ellipsoid_fig0}
% \end{subfigure}\,\, %
% \begin{subfigure}{.24\textwidth} %
% \centering
% \includegraphics[width=\textwidth]{fig/rotating_circle_ellipsoid_fig1}
% \caption{}
% \label{fig:rotating_circle_ellipsoid_fig1}
% \end{subfigure}
% \begin{subfigure}{.24\textwidth} %
% \centering
% \includegraphics[width=\textwidth]{fig/rotating_circle_ellipsoid_fig2}
% \caption{}
% \label{fig:rotating_circle_ellipsoid_fig2}
% \end{subfigure}
% \begin{subfigure}{.24\textwidth} %
% \centering
% \includegraphics[width=\textwidth]{fig/rotating_circle_ellipsoid_fig3}
% \caption{}
% \label{fig:rotating_circle_ellipsoid_fig3}
% \end{subfigure}

% \begin{subfigure}{.24\textwidth} %
% \centering
% \includegraphics[width=\textwidth]{fig/rotating_circle_fluid_fig0}
% \caption{}
% \label{fig:rotating_circle_fluid_fig0}
% \end{subfigure}\,\, %
% \begin{subfigure}{.24\textwidth} %
% \centering
% \includegraphics[width=\textwidth]{fig/rotating_circle_fluid_fig1}
% \caption{}
% \label{fig:rotating_circle_fluid_fig1}
% \end{subfigure}
% \begin{subfigure}{.24\textwidth} %
% \centering
% \includegraphics[width=\textwidth]{fig/rotating_circle_fluid_fig2}
% \caption{}
% \label{fig:rotating_circle_fluid_fig2}
% \end{subfigure}
% \begin{subfigure}{.24\textwidth} %
% \centering
% \includegraphics[width=\textwidth]{fig/rotating_circle_fluid_fig3}
% \caption{}
% \label{fig:rotating_circle_fluid_fig3}
% \end{subfigure}

% \begin{subfigure}{.24\textwidth} %
% \centering
% \includegraphics[width=\textwidth]{fig/rotating_circle_rotation_fig0}
% \caption{}
% \label{fig:rotating_circle_rotation_fig0}
% \end{subfigure}\,\, %
% \begin{subfigure}{.24\textwidth} %
% \centering
% \includegraphics[width=\textwidth]{fig/rotating_circle_rotation_fig1}
% \caption{}
% \label{fig:rotating_circle_rotation_fig1}
% \end{subfigure}
% \begin{subfigure}{.24\textwidth} %
% \centering
% \includegraphics[width=\textwidth]{fig/rotating_circle_rotation_fig2}
% \caption{}
% \label{fig:rotating_circle_rotation_fig2}
% \end{subfigure}
% \begin{subfigure}{.24\textwidth} %
% \centering
% \includegraphics[width=\textwidth]{fig/rotating_circle_rotation_fig2}
% \caption{}
% \label{fig:rotating_circle_rotation_fig2}
% \end{subfigure}
% \caption{Object avoidance of a rotating ellipse obstacle by a linear attractor DS with DMM (e - h), IFD (i - l) and LRS (m - p) compared to the dynamical system without any objects (a - d)}
% \label{fig:rotating_circle_fig}
% \end{figure}

%%% Local Variables:
%%% mode: latex
%%% TeX-master: "../main"
%%% End:


\subsubsection{Quantitative comparison}
For the quantitative comparison, a set of initial conditions is chosen and the simulation is run with either IFD or DMM until convergence or the time limit is reached. \\ %
According to the SRC measure the IFD modifies the DS less (Tab.~\ref{tab:twoMovingRotatingObs} -\ref{tab:fastMovingEllipse}), as it has a lower value of at least 15\%  compared to the DMM. \\
The IFD tends to have more collisions, with up to 13\% of the trajectories colliding, while the DMM never collides with an obstacle (Tab.~\ref{tab:fastMovingEllipse}). \\
The convergence time $T_{conv}$ is in the same range for IFD and DMM (Tab.~\ref{tab:fourStaticObjects}).The path length $D_{conv}$ and the relative convergence energy $\hat E_{conv}$ on the other hand are lower for the IFD during the three simulations. The DMM slightly lags having around 10\%-20\% higher convergence distance and up to 100\% higher relative convergence energy. In general, the energy seems to correlate with the value of the SRC. \\
The computational time for the IFD and the DMM are in the same range and vary only by about 10\%: the IFD has a higher computational time with several obstacles.  \\
The complexity based on the length of the code is comparable for both algorithms with around 160 ($\pm$20) lines.

\begin{table}[h]
\centering
\caption{Simulation metrics of a dynamical system with two moving obstacles with time step $dt = 0.005 s$, maximum iteration $i_{max} = 1200$ (similar to Fig.~\ref{fig:twoMovingRotatingObs_DMM} \& \ref{fig:twoMovingRotatingObs_IFD}).}
\label{tab:twoMovingRotatingObs}
\begin{tabular}{|l|r|r|r|} \hline
 &\multicolumn{1}{c|}{-} &\multicolumn{1}{c|}{DMM} &\multicolumn{1}{c|}{IFD} \\ \hline
SRC $[m2/s2]$ &0.0000 &197.6862 &166.6479 \\ \hline
$N_{pen} \, [\%]$ &0.0 &0.0 &0.0 \\ \hline
$T_{pen} []$ &0 &0 &0 \\ \hline
$T_{conv} [s]$ &5.3050 &5.9900 &5.9900 \\ \hline
$D_{conv} [m]$ &22.1793 &25.7054 &24.0317 \\ \hline
$\hat E_{conv} [J/kg]$ &25.3305 &87.8503 &61.5909\\ \hline
$T_{comp} [ms]$ &0.0000 &0.0113 &0.0116 \\ \hline
\end{tabular}
\end{table}


\begin{table}[h]
\centering
\caption{Simulation metrics of a dynamical system with four static objects  with time step $dt = 0.01 s$, maximum iteration $i_{max} = 1200$ and convergence tolerance $r_\epsilon = 0.1 m$ (similar to Fig.~\ref{fig:singleAttractor_severalObstacles_time}).}
\label{tab:fourStaticObjects}
\begin{tabular}{|l|r|r|r|} \hline
 &\multicolumn{1}{c|}{-} &\multicolumn{1}{c|}{DMM} &\multicolumn{1}{c|}{IFD} \\ \hline
SRC $[m2/s2]$ &0.0000 &129.2555 &99.4898 \\ \hline
$N_{pen} \; [\%]$ &0.0 &0.0 &3.3 \\ \hline
$T_{pen} [-] $ &0 &0 &2 \\ \hline
$T_{conv} [s]$ &4.6250 &9.6500 &9.7750 \\ \hline
$D_{conv} [m]$ &9.6599 &12.1595 &11.6221 \\ \hline
$\hat E_{conv} [J/kg]$ &37.0222 &127.2202 &55.9537 \\ \hline
$T_{comp} [ms]$ &0.0000 &0.0096 &0.0102 \\ \hline
\end{tabular}
\end{table}



\begin{table}[h]
\centering
\caption{Simulation metrics of a dynamical system with a fast moving ellipse (similar to Fig.~\ref{fig:fastMovingEllipse_fig}) with time step $dt = 0.005 s$, maximum iteration $i_{max} = 1200$ and convergence tolerance $r_\epsilon = 0.1 m$ .}
\label{tab:fastMovingEllipse}
\begin{tabular}{|l|r|r|r|} \hline
 &\multicolumn{1}{c|}{-} &\multicolumn{1}{c|}{DMM} &\multicolumn{1}{c|}{IFD} \\ \hline
SRC $[m2/s2]$ &0.00 &665.40 &322.58 \\ \hline
$N_{pen} \, [\%]$ &0.0 &0.0 &13.3\\ \hline
$T_{pen} [-]$ &0 &0 &54 \\ \hline
$T_{conv} [s]$ &5.3050 &5.9900 &5.8150 \\ \hline
$D_{conv} [m]$ &22.2363 &31.9179 &26.5775 \\ \hline
$\hat E_{conv} [J/kg]$ &19.1033 &175.4631 &109.4048 \\ \hline
$T_{comp} [ms]$ &0.0000 &0.0079 &0.0072 \\ \hline
\end{tabular}
\end{table}

% \begin{table}[h]
% \centering
% \caption{Complexity of the algorithm is analyzed as the number of lines of acting code (without comments and backspaces) used to express the obstacle avoidance. }
% \label{tab:computationalComplexity}
% \begin{tabular}{|l|r|} \hline
% & Length of code [\# of lines]\\ \hline
% DMM & 143 \\ \hline
% IFD & 183 \\ \hline
% LRS & 169 \\ \hline
% \end{tabular}
% \end{table}




%%% Local Variables:
%%% mode: latex
%%% TeX-master: "../main"
%%% End:


%%% Local Variables:
%%% mode: latex
%%% TeX-master: "../main"
%%% End:
